%===============================================================================
% LaTeX sjabloon voor de bachelorproef toegepaste informatica aan HOGENT
% Meer info op https://github.com/HoGentTIN/latex-hogent-report
%===============================================================================

\documentclass[dutch,dit,thesis]{hogentreport}

% TODO:
% - If necessary, replace the option `dit`' with your own department!
%   Valid entries are dbo, dbt, dgz, dit, dlo, dog, dsa, soa
% - If you write your thesis in English (remark: only possible after getting
%   explicit approval!), remove the option "dutch," or replace with "english".

\usepackage{lipsum} % For blind text, can be removed after adding actual content

%% Pictures to include in the text can be put in the graphics/ folder
\graphicspath{{../graphics/}}

%% For source code highlighting, requires pygments to be installed
%% Compile with the -shell-escape flag!
%% \usepackage[chapter]{minted}
%% If you compile with the make_thesis.{bat,sh} script, use the following
%% import instead:
\usepackage[chapter,outputdir=../output]{minted}
\usemintedstyle{solarized-light}

%% Formatting for minted environments.
\setminted{%
    autogobble,
    frame=lines,
    breaklines,
    linenos,
    tabsize=4
}

%% Ensure the list of listings is in the table of contents
\renewcommand\listoflistingscaption{%
    \IfLanguageName{dutch}{Lijst van codefragmenten}{List of listings}
}
\renewcommand\listingscaption{%
    \IfLanguageName{dutch}{Codefragment}{Listing}
}
\renewcommand*\listoflistings{%
    \cleardoublepage\phantomsection\addcontentsline{toc}{chapter}{\listoflistingscaption}%
    \listof{listing}{\listoflistingscaption}%
}

% Other packages not already included can be imported here

%%---------- Document metadata -------------------------------------------------
\author{Wauters Sander}
\supervisor{Mevr. I. Malfait}
\cosupervisor{Dhr. P. De Seranno}
\title{Proof of concept: De update automatiseren van Angular versie 16 naar versie 20 in de applicaties van een end-to-end kredietdienstverlener.}
% \academicyear{\advance\year by -1 \the\year--\advance\year by 1 \the\year}
\academicyear{2025--2026}
\examperiod{1}
\degreesought{\IfLanguageName{dutch}{Professionele bachelor in de toegepaste informatica}{Bachelor of applied computer science}}
\partialthesis{false} %% To display 'in partial fulfilment'
%\institution{Internshipcompany BVBA.}

%% Add global exceptions to the hyphenation here
\hyphenation{back-slash}

%% The bibliography (style and settings are  found in hogentthesis.cls)
\addbibresource{bachproef.bib}            %% Bibliography file
\addbibresource{../voorstel/voorstel.bib} %% Bibliography research proposal
\defbibheading{bibempty}{}

%% Prevent empty pages for right-handed chapter starts in twoside mode
\renewcommand{\cleardoublepage}{\clearpage}

\renewcommand{\arraystretch}{1.2}

%% Content starts here.
\begin{document}

%---------- Front matter -------------------------------------------------------

\frontmatter

\hypersetup{pageanchor=false} %% Disable page numbering references
%% Render a Dutch outer title page if the main language is English
\IfLanguageName{english}{%
    %% If necessary, information can be changed here
    \degreesought{Professionele Bachelor toegepaste informatica}%
    \begin{otherlanguage}{dutch}%
       \maketitle%
    \end{otherlanguage}%
}{}

%% Generates title page content
\maketitle
\hypersetup{pageanchor=true}

%%=============================================================================
%% Voorwoord
%%=============================================================================

\chapter*{\IfLanguageName{dutch}{Woord vooraf}{Preface}}%
\label{ch:voorwoord}

%% TODO:
%% Het voorwoord is het enige deel van de bachelorproef waar je vanuit je
%% eigen standpunt (``ik-vorm'') mag schrijven. Je kan hier bv. motiveren
%% waarom jij het onderwerp wil bespreken.
%% Vergeet ook niet te bedanken wie je geholpen/gesteund/... heeft

Deze bachelorproef vormt het sluitstuk van mijn opleiding Toegepaste Informatica, met als specialisatie Mobile en Enterprise Development.
Tijdens deze opleiding kreeg ik de kans om aan verschillende projecten te werken en deed ik ervaring op met verschillende programmeertalen en frameworks. 
Wat mij daarbij het meest boeide, was het ontwikkelingsproces zelf. 
Om die reden koos ik ervoor om mijn bachelorproef uit te werken rond een technisch onderwerp.
\medskip

De originele probleemstelling komt van mijn co-promotor, Peter De Seranno.
Het probleem was dat er verschillende Angular-applicaties geüpdatet moesten worden naar een nieuwe versie.
Wat mij opviel, was het grote aantal herhalingen in dit updateproces.
Processen die zich herhalen zijn geschikt om te automatiseren.
Zo ontstond het idee om software te ontwikkelen die het updateproces kan ondersteunen.
Dit project was een leerrijke ervaring waarin ik mijn technische kennis kon toepassen. 
Verder has het Angular framework nieuw voor mij.
Het was interessant om hier meer over te leren.
\medskip

Voor dit project kon ik rekenen op de hulp en begeleiding van enkele personen die ik graag wil bedanken.
In de eerste plaats wil ik mijn promotor, mevrouw Irina Malfait, bedanken voor de opvolging en begeleiding van mijn werk. 
Haar gerichte en constructieve feedback hielp mij om deze bachelorproef naar een hoger niveau te brengen.
Daarnaast wil ik mijn oprechte dank uitspreken aan mijn co-promotor, Peter De Seranno, voor zijn technische ondersteuning en het nalezen van mijn werk. 
Zijn input zorgde ervoor dat ik het overzicht behield en dat het onderzoek steeds praktisch en relevant bleef.
\medskip

Tot slot wil ik mijn ouders bedanken voor hun voortdurende steun en begrip. 
Dankzij hun aanmoediging kon ik de overstap maken naar deze studierichting en mijn opleiding met vertrouwen verderzetten.
Met deze bachelorproef hoop ik iets bij te dragen aan het ontwikkelingsproces en de manier waarop we software refactoren en onderhouden.

%%=============================================================================
%% Samenvatting
%%=============================================================================

% TODO: De "abstract" of samenvatting is een kernachtige (~ 1 blz. voor een
% thesis) synthese van het document.
%
% Een goede abstract biedt een kernachtig antwoord op volgende vragen:
%
% 1. Waarover gaat de bachelorproef?
% 2. Waarom heb je er over geschreven?
% 3. Hoe heb je het onderzoek uitgevoerd?
% 4. Wat waren de resultaten? Wat blijkt uit je onderzoek?
% 5. Wat betekenen je resultaten? Wat is de relevantie voor het werkveld?
%
% Daarom bestaat een abstract uit volgende componenten:
%
% - inleiding + kaderen thema
% - probleemstelling
% - (centrale) onderzoeksvraag
% - onderzoeksdoelstelling
% - methodologie
% - resultaten (beperk tot de belangrijkste, relevant voor de onderzoeksvraag)
% - conclusies, aanbevelingen, beperkingen
%
% LET OP! Een samenvatting is GEEN voorwoord!

%%---------- Nederlandse samenvatting -----------------------------------------
%
% TODO: Als je je bachelorproef in het Engels schrijft, moet je eerst een
% Nederlandse samenvatting invoegen. Haal daarvoor onderstaande code uit
% commentaar.
% Wie zijn bachelorproef in het Nederlands schrijft, kan dit negeren, de inhoud
% wordt niet in het document ingevoegd.

% \IfLanguageName{english}{%
% \selectlanguage{dutch}
% \chapter*{Samenvatting}
% \lipsum[1-4]
% \selectlanguage{english}
% }{}

%%---------- Samenvatting -----------------------------------------------------
% De samenvatting in de hoofdtaal van het document

\chapter*{\IfLanguageName{dutch}{Samenvatting}{Abstract}}

% - inleiding + kaderen thema
Het Angular-framework vereenvoudigt het ontwikkelingsproces voor het bouwen van dynamische webapplicaties.
Zoals bij de meeste software ontvangt Angular regelmatig updates.
Deze updates zijn noodzakelijk, omdat ze de cyberveiligheid verbeteren.
Het toepassen van dergelijke updates is echter niet altijd vanzelfsprekend.
% - probleemstelling
Angular verwijdert verouderde functionaliteiten uit het framework.
Daardoor moet de broncode van Angular-applicaties aangepast worden.
Dit type updates vindt om de 6 maanden plaats.
Bij meerdere enterprise-applicaties kan de benodigde tijd voor dit updateproces snel oplopen.
Het bedrijf Stater ervaart dit probleem.
Stater is een end-to-end dienstverlener voor zowel hypothecaire als consumentenkredieten.
Zij willen al hun applicaties updaten van Angular v16 naar v20.
De sprong van vier versies betekent dat er vermoedelijk veel aanpassingen in de broncode nodig zijn.
\medskip

% - (centrale) onderzoeksvraag
Om deze reden wil dit onderzoek achterhalen in welke mate de automatisering van het updateproces van Angular v16 naar v20, over meerdere applicaties, de onderhoudstijd voor ontwikkelaars kan verlagen.
% - onderzoeksdoelstelling
Om hierop een antwoord te formuleren, ontwikkelt dit onderzoek een applicatie om het updateproces tee ondersteunen.
Deze applicatie, de \emph{updater}, fungeert als proof of concept.
% - methodologie
Voor de implementatie van de updater maken we gebruik van zoek- en vervangfuncties in combinatie met de TypeScript Compiler API.
De updater kan programmatisch geconfigureerd worden.
Dit biedt aanzienlijke flexibiliteit, waardoor de updater ook bij toekomstige aanpassingen inzetbaar blijft.
\medskip

% - resultaten (beperk tot de belangrijkste, relevant voor de onderzoeksvraag)
Met onze aanpak is het mogelijk om een kwart van alle nodigde wijzigingen in de update van v16 naar v20 te automatiseren.
Daarnaast zijn we niet beperkt tot aanpassingen in Angular.
Onze aanpak werkt op alle TypeScript-code binnen een project.
% - conclusies, aanbevelingen, beperkingen
De voornaamste beperking van de updater is het uitvoeren van wijzigingen aan logic.
Aanpassingen die inzicht vereisen in de werking van de applicatie vormen hierbij een uitdaging.
Om de updater optimaal te benutten, is het aangewezen om enkel syntactische aanpassingen te automatiseren.



%---------- Inhoud, lijst figuren, ... -----------------------------------------

\tableofcontents

% In a list of figures, the complete caption will be included. To prevent this,
% ALWAYS add a short description in the caption!
%
%  \caption[short description]{elaborate description}
%
% If you do, only the short description will be used in the list of figures

\listoffigures

% If you included tables and/or source code listings, uncomment the appropriate
% lines.
\listoftables

\listoflistings

% Als je een lijst van afkortingen of termen wil toevoegen, dan hoort die
% hier thuis. Gebruik bijvoorbeeld de ``glossaries'' package.
% https://www.overleaf.com/learn/latex/Glossaries

%---------- Kern ---------------------------------------------------------------

\mainmatter{}

% De eerste hoofdstukken van een bachelorproef zijn meestal een inleiding op
% het onderwerp, literatuurstudie en verantwoording methodologie.
% Aarzel niet om een meer beschrijvende titel aan deze hoofdstukken te geven of
% om bijvoorbeeld de inleiding en/of stand van zaken over meerdere hoofdstukken
% te verspreiden!

%%=============================================================================
%% Inleiding
%%=============================================================================

\chapter{\IfLanguageName{dutch}{Inleiding}{Introduction}}%
\label{ch:inleiding}

% Waar gaat het onderzoek over?
Software frameworks zoals Angular vereenvoudigen het maken van dynamische web applicaties.
Zoals vele software krijgt Angular geregeld updates.
Deze updates komen met verschillende voordelen, zoals: nieuwe functionaliteiten, betere performanie, bug fixes, \dots.
Het toepassen van deze updates is niet altijd even vanzelfsprekend.
Soms dienen nieuwe functionaliteiten als vervanging op oudere.
Dit zort ervoor dat de code die Angular aanspreek ook moet veranderen.

% Wat is de context?
Het bedrijf Stater is een end-to-end dienstverlener voor zowel hypothecaire als consumentenkredieten.
Ze ondersteunen de kredietverstrekker voor de dienstverlening aan consumenten.
Intern zijn er meerder applicatie die gebruik maken van het Angular framework.
Specifiek Angular versie 16 (v16).
Stater zou graag al deze applicaties updaten naar de meest recente versie, Angular versie 20 (v20).

\section{\IfLanguageName{dutch}{Probleemstelling}{Problem Statement}}%
\label{sec:probleemstelling}

% Wat is het probleem?
De sprong van 4 versies betekend welicht dat er veel aanpassingen aan de broncodee nodig zijn.
Dit probleem vermeenigvuldicht zich met de grote van de broncode en het aantal applicaties dat deze updates nodig hebben.
Het manueel uitvoeren van al deze veranderingen neemt veel tijd in beslag.
Dit probleem is niet eenmalig, Angular krijgt een nieuwe versie om de 6 maanden.

% Waarom is dit een probleem?
De onderhoudskost van software kan snel tot boven de 60\% van de totale project kost oplopen.
Dit soort onderhoud kan ook niet eewig uitgesteld worden.
Software updates zijn noodzakelijk om de cyberveiligheid van een applicatie te garanderen.

% Voor wie is dit een probleem?
Om deze redenen is het vereenvoudigen van het update process best interesant.
Voor de programmeurs dat de updates toepassen verminderd dit de werkdruk.
Voor het bedrijf Stater betekend dit dat de onderhoudstijd, en bij gevolg kost, lager kan liggen.

\section{\IfLanguageName{dutch}{Onderzoeksvraag}{Research question}}%
\label{sec:onderzoeksvraag}

% Wat is de onderzoeksvraag?
Op basis van de bovenstaande probleemstelling is de volgende onderzoeksvraag gefromuleerd: in welke mate kan de automatisering van het updateproces van Angular v16 naar v20, bij meerdere applicaties, de onderhoudstijd voor de ontwikkelaars verlagen?

% Wat zijn de deelvragen?
Om deze onderzoeksvraag te beantwoorden zijn volgende deelvragen opgesteld:
\begin{itemize}
  \item Hoeveel veranderingen moeten uitgevoerd worden om Angular van v16 naar v20 te updaten?
  \item Welke manieren bestaan om code automatisch aan te passen zonder ongewenste veranderingen uit te voeren?
  \item Welke manier om code automatisch aan te passen is het meest geschikt om in deze casus toe te passen?
  \item Wat zijn statistisch gezien de meest voorkomende problemen bij het updaten van code?
\end{itemize}

\section{\IfLanguageName{dutch}{Onderzoeksdoelstelling}{Research objective}}%
\label{sec:onderzoeksdoelstelling}

% Wat probeert het onderzoek te bereiken?
Om de onderzoeksvraag te beantwoorden wordt als proof of concept een applicatie ontwikkeld dat de programmeurs ondersteunt in het updateprocess.
In de rest van dit onderzoek zal naar deze applicatie verwezen worden als de ``updater''.

% Wat zijn de criteria voor succes?
De updater moet minstens voldoen aan de volgende criteria:

\begin{itemize}
  \item De updater is van de ontwikkelaars voor de ontwikkelaars. De bedoeling is dat de persoon die de update uitvoerd de updater kan instellen.
  \item De updater moet passen in de huidige workflow van de ontwikkelaars.
  \item De updater moet aanpasbaar zijn aan de uit te voeren update. Het moet kunnen gebruikt worden bij de volgende update.
  \item De updater mag geen nieuwe bugs introduceren. Gegeven dat de instelling correct is mag het geen fouten maken.
  \item De updater mag niet gekoppeld zijn aan Angular. Dit zorgt ervoor dat de updater zelf niet geupdate moet worden bij een nieuwe Angular versie.
\end{itemize}

\section{\IfLanguageName{dutch}{Opzet van deze bachelorproef}{Structure of this bachelor thesis}}%
\label{sec:opzet-bachelorproef}

% Hoe is de rest van de tekst opgebouwd?
De rest van deze bachelorproef is als volgt opgebouwd:

In Hoofdstuk~\ref{ch:stand-van-zaken} wordt een overzicht gegeven van de stand van zaken binnen het onderzoeksdomein, op basis van een literatuurstudie.

In Hoofdstuk~\ref{ch:methodologie} wordt de methodologie toegelicht en worden de gebruikte onderzoekstechnieken besproken om een antwoord te kunnen formuleren op de onderzoeksvragen.

In Hoofdstuk~\ref{ch:conclusie}, tenslotte, wordt de conclusie gegeven en een antwoord geformuleerd op de onderzoeksvragen. Daarbij wordt ook een aanzet gegeven voor toekomstig onderzoek binnen dit domein.

\chapter{\IfLanguageName{dutch}{Stand van zaken}{State of the art}}%
\label{ch:stand-van-zaken}

% Welke termen en technologiën moet iemand kennen voor het onderzoek te snappen?
In dit hoofdstuk bespreken we de verschillende technologiën dat betrekking hebben tot dit onderzoek.
Deze literatuurstudie start met een omschrijven van het Angular framework en hoe een Angular project gestructureerd is.
Vervolgens wordt uitleg gegeven over de TypeScript programmeertaal, specifiek hoe Angular dit gebruikt.
Ten slotte volgt een overzicht van verschillende gekende manieren om code automatisch aan te passen.

\section{Angular}
\label{ch:stand-van-zaken:angular}

% Wat is Angular?
Angular, ook wel Angular2 genoemd, is een user interface (UI) framework ontwikkeld door Google in 2016 \autocite{Cincovic2019}.
Het is gratis, open-source en wordt onderhouden door diverse groep van ontwikkelaars.
% Waarvoor dient het?
Angular wordt gebruikt voor het maken van single-page web applicaties dat zowel client als server side rendered kunnen worden.
% Hoe maakt het een UI?
Voor het opbouwen van een UI in Angular worden ``componenten'' gebruikt.
% Wat is een component in Angular?
Een component binnen Angular wordt door \textcite{Kaufman2016} omschreven als een zelfstandige en herbruikbare bouwblok.
Componenten encapsuleren de bedrijfslogica, structuur een stijl van een deel van de UI.
Het combineren van verschillende componenten laat toe om complexte UI's te maken.

% Hoe werkt het?
Angular is een opinionated framework.
\textcite{Parker2017} defineert een framework als opinionated als het de ontwikkelaar aanstuurt om op een specifieke manier te te werken.
Opiniononated frameworks houden zich aan stricte conventies dat dicteren hoe een project is opgesteld en geschreven.
% Wat kan het?
Het Angular framework komt ingebouwd met verschillende functionaliteiten dat de ontwikkeling van een applicatie aanstuurt \autocite{Wilken2018}.
Zoals eerder besproken maakt Angular gebruik van componenten voor het bouwen van een UI.
Verder komt het met functies dat toelaten om unit testen te schrijven voor deze componenten.
Angular heeft een collectie van command line (CLI) tools dat de ontwikkelaars helpt bij het maken van een applicatie, bijvoorbeel het genereren van een blanko component met bijhorende testen in één commando.
Verder komt het met een eigen Hyper Text Transfer Protocol (HTTP) client voor een applicatie te verbinden met een backend service over het internet.

% Hoe is een Angular project gestructureerd?
Angular is gebaseerd op TypeScript en gebruikt dit in combinatie met andere technologiën.
In een Angular project zijn de volgende bestanden terug te vinden:
\begin{itemize}
  \item TypeScript, de TypeScript programmeertaal wordt gebruikt voor de implementatie van de bedrijfslogica en testen.
  \item HTML, HTML wordt bebruikt voor de achterliggende structuur van de UI te omschrijven. 
    In de context van Angular componenten wordt hiernaar verwezen als een ``template''.
  \item CSS, CSS wordt gebruikt om de visuele representatie van de UI te omschrijven
  \item JSON, JSON wordt gebruikt voor het configureren van Angular en TypeScript.
\end{itemize}

\section{TypeScript}
\label{ch:stand-van-zaken:typescript}

% Wat is TypeScript?

- Basically JavaScript met een type system.
- Compileerd naar JavaScript.
- Inheritly object oriented, basically the prototype patern on steroids.

% Hoe wordt HTML en CSS binnen Angular gebruikt?

- HTML kan in apparte bestanden of als een string in een TypeScript component (templates) in een decorator.
- CSS in apparte bestanden en wordt gelinkt in TypeScript via een decorator.

\section{Automatisch refactoren}
\label{ch:stand-van-zaken:refactoring}

% Wat is refactoren?

- De source code van een applicatie aanpassen om de operatie aan te passen zonder de functionaliteit te veranderen.

% Welke manieren bestaan er?

- De ander hoofdstukken omschrijven de verschillende manieren dat van toepassing zijn.

\subsection{Find \& replace}
\label{ch:stand-van-zaken:refactoring:find-and-replace}

% Wat is find & replace?

- Gebaseerd op tekst of Regex.

% Waarom is dit relevant?

- De simpelste vorm om in bulk code aan te passen.

% Hoe werkt dit?

- Pattern matching.
- Verschillende algorithme.

% Wat zijn de voordelen?

- Simpel te begrijpen.
- Simpel te implementeren.

% Wat zijn de nadelen?

- Geen vat op syntax.
- Geen vat op semantiek.

\subsection{Compiler gebaseerd}
\label{ch:stand-van-zaken:refactoring:compiler}

% Wat is compiler gebaseerd refactoren?

- De functionaliteit van de compiler gebruiken om code aan te passen.

% Waarom is dit relevant?

- TypeScript is compiled.
- TypeScript heeft een compiler API.

% Hoe werkt dit?

- Compiler leest de code in als een Abstract Syntax Tree (AST).

% Wat zijn de voordelen?

- Bestaande API.
- Heeft vat op syntax.
- Goed gedocumenteerd.

% Wat zijn de nadelen?

- Geen vat op semantiek.

\subsection{Language Server Protocol (LSP) gebaseerd}
\label{ch:stand-van-zaken:refactoring:lsp}

% Wat is het LSP gebaseerd refactoren?

- LSP is de technologie achter de refactoring tools in meeste moderen IDE's en text editors.

% Waarom is dit relevant?

- LSP is de defactor standaard.

% Wat kan de TypeScript LSP?
% Wat kan de Angular LSP?
% Wat zijn de voordelen?

- Heeft vat op syntax.

% Wat zijn de nadelen?

- Geen vat op semantiek.
- Weinig tot geen documentatie tot de interne werking.
- Wordt zelden tot nooit programatisch aangesproken.

\subsection{Artificiele Inteligentie (AI) gebaseerd}
\label{ch:stand-van-zaken:refactoring:ai}

% Wat is AI gebaseerd refactoren?

- AI de code laten inlezen en veranderingen laten toebrengen.

% Waarom is dit relevant?

- AI is overal vandaag.

% Hoe werkt dit?

- No one knows exectly, it's a black box.
- Geeft statistisch gezien het beste antwoord op een vraag op basis van gekende data.

% Wat zijn de voordelen?

- Kan vat hebben op syntax.
- Kan vat hebben op semantiek.

% Wat zijn de nadelen?

- Voor het maken van een AI is een grote dataset nodig.
- Geen absolute zekerheid of het een corecte output zal geven.
- Open AI tools zoals ChatGPT op interne code gebruiken geeft problemen met confidentialiteit.

\subsection{Gekende problemen}
\label{ch:stand-van-zaken:refactoring:known-problems}

% Wat zijn de gekende problemen bij refactoren?

- Kan soms nieuwe bugs introduceren.







% Tip: Begin elk hoofdstuk met een paragraaf inleiding die beschrijft hoe
% dit hoofdstuk past binnen het geheel van de bachelorproef. Geef in het
% bijzonder aan wat de link is met het vorige en volgende hoofdstuk.

% Pas na deze inleidende paragraaf komt de eerste sectiehoofding.

% Dit hoofdstuk bevat je literatuurstudie. De inhoud gaat verder op de inleiding, maar zal het onderwerp van de bachelorproef *diepgaand* uitspitten. De bedoeling is dat de lezer na lezing van dit hoofdstuk helemaal op de hoogte is van de huidige stand van zaken (state-of-the-art) in het onderzoeksdomein. Iemand die niet vertrouwd is met het onderwerp, weet nu voldoende om de rest van het verhaal te kunnen volgen, zonder dat die er nog andere informatie moet over opzoeken \autocite{Pollefliet2011}.
% 
% Je verwijst bij elke bewering die je doet, vakterm die je introduceert, enz.\ naar je bronnen. In \LaTeX{} kan dat met het commando \texttt{$\backslash${textcite\{\}}} of \texttt{$\backslash${autocite\{\}}}. Als argument van het commando geef je de ``sleutel'' van een ``record'' in een bibliografische databank in het Bib\LaTeX{}-formaat (een tekstbestand). Als je expliciet naar de auteur verwijst in de zin (narratieve referentie), gebruik je \texttt{$\backslash${}textcite\{\}}. Soms is de auteursnaam niet expliciet een onderdeel van de zin, dan gebruik je \texttt{$\backslash${}autocite\{\}} (referentie tussen haakjes). Dit gebruik je bv.~bij een citaat, of om in het bijschrift van een overgenomen afbeelding, broncode, tabel, enz. te verwijzen naar de bron. In de volgende paragraaf een voorbeeld van elk.
% 
% \textcite{Knuth1998} schreef een van de standaardwerken over sorteer- en zoekalgoritmen. Experten zijn het erover eens dat cloud computing een interessante opportuniteit vormen, zowel voor gebruikers als voor dienstverleners op vlak van informatietechnologie~\autocite{Creeger2009}.
% 
% Let er ook op: het \texttt{cite}-commando voor de punt, dus binnen de zin. Je verwijst meteen naar een bron in de eerste zin die erop gebaseerd is, dus niet pas op het einde van een paragraaf.
% 
% \begin{figure}
%   \centering
%   \includegraphics[width=0.8\textwidth]{grail.jpg}
%   \caption[Voorbeeld figuur.]{\label{fig:grail}Voorbeeld van invoegen van een figuur. Zorg altijd voor een uitgebreid bijschrift dat de figuur volledig beschrijft zonder in de tekst te moeten gaan zoeken. Vergeet ook je bronvermelding niet!}
% \end{figure}
% 
% \begin{listing}
%   \begin{minted}{python}
%     import pandas as pd
%     import seaborn as sns
% 
%     penguins = sns.load_dataset('penguins')
%     sns.relplot(data=penguins, x="flipper_length_mm", y="bill_length_mm", hue="species")
%   \end{minted}
%   \caption[Voorbeeld codefragment]{Voorbeeld van het invoegen van een codefragment.}
% \end{listing}
% 
% \lipsum[7-20]
% 
% \begin{table}
%   \centering
%   \begin{tabular}{lcr}
%     \toprule
%     \textbf{Kolom 1} & \textbf{Kolom 2} & \textbf{Kolom 3} \\
%     $\alpha$         & $\beta$          & $\gamma$         \\
%     \midrule
%     A                & 10.230           & a                \\
%     B                & 45.678           & b                \\
%     C                & 99.987           & c                \\
%     \bottomrule
%   \end{tabular}
%   \caption[Voorbeeld tabel]{\label{tab:example}Voorbeeld van een tabel.}
% \end{table}


%%=============================================================================
%% Methodologie
%%=============================================================================

\chapter{\IfLanguageName{dutch}{Methodologie}{Methodology}}%
\label{ch:methodologie}

%% TODO: In dit hoofstuk geef je een korte toelichting over hoe je te werk bent
%% gegaan. Verdeel je onderzoek in grote fasen, en licht in elke fase toe wat
%% de doelstelling was, welke deliverables daar uit gekomen zijn, en welke
%% onderzoeksmethoden je daarbij toegepast hebt. Verantwoord waarom je
%% op deze manier te werk gegaan bent.
%% 
%% Voorbeelden van zulke fasen zijn: literatuurstudie, opstellen van een
%% requirements-analyse, opstellen long-list (bij vergelijkende studie),
%% selectie van geschikte tools (bij vergelijkende studie, "short-list"),
%% opzetten testopstelling/PoC, uitvoeren testen en verzamelen
%% van resultaten, analyse van resultaten, ...
%%
%% !!!!! LET OP !!!!!
%%
%% Het is uitdrukkelijk NIET de bedoeling dat je het grootste deel van de corpus
%% van je bachelorproef in dit hoofstuk verwerkt! Dit hoofdstuk is eerder een
%% kort overzicht van je plan van aanpak.
%%
%% Maak voor elke fase (behalve het literatuuronderzoek) een NIEUW HOOFDSTUK aan
%% en geef het een gepaste titel.

% Wat gebeurt er in dit hoofdstuk?
In dit hoofdstuk maakt het onderzoek een beslissing over hoe de updater ontwikkeld wordt.
Deze beslissing is gebaseerd op de huidige stand van zaken en de noden van het bedrijf, zoals besproken in Hoofdstuk~\ref{sec:onderzoeksdoelstelling}.
We bespreken de verschillende soorten aanpassingen die moeten gebeuren om een Angular applicatie van v16 naar v20 te updaten.
Vervolgens bespreken we de proof of concept.
Hoe de updater geëvalueerd zal worden aan de hand van een testomgeving.
Wat de updater kan en doet.
En tenslotte bespreken we de evaluatie van de updater.
In Hoofdstuk~\ref{ch:proof-of-concept} wordt de technische uitwerking van de updater in meer detail besproken.

\section{Plan van aanpak}
\label{ch:plan-van-aanpak}

% Hoe ziet de updater eruit?
Op basis van de huidige stand van zaken en de noden van het bedrijf, kiest het onderzoek voor het volgende plan van aanpak.
We ontwikkelen een collectie aan helperfuncties om in TypeScript een command line applicatie te maken.
Deze applicatie maakt onderliggend gebruik van zoek- en vervangfuncties op basis van regex in combinatie met de TypeScript compiler API.
Met de hulp van deze functies kunnen we programmatisch een updater ontwerpen.

% Waarom een CLI-applicatie?
Angular komt reeds met CLI-tools; door de updater een CLI-applicatie te maken, past het in de huidige workflow.
Verder geeft dit de mogelijkheid om alle commando's samen te voegen in en script om de updater op meerdere projecten te laten uitvoeren.

% Waarom een collectie van helperfuncties?
De nodige aanpassingen aan een applicatie zijn afhankelijk van de Angular versie.
Dit maakt het praktisch onmogelijk om een ``one size fits all'' updater te maken die werkt op toekomstige versies.
Voor de nieuwe versie zal dan ook een nieuwe updater geconfigureerd moeten worden.
Door de updater programmatisch te configureren, is er een hoge flexibiliteit en uitbreidbaarheid.
De helperfuncties zorgen voor een extra abstractielaag, wat toelaat om snel nieuwe updates te automatiseren.
Als programmeertaal van de updater is voor TypeScript gekozen.
Omdat dit dezelfde taal is die Angular gebruikt en per extensie de taal waar de ontwikkelaars bekend mee zijn.

% Waarom TypeScript compiler API?
De implementatie van de helperfuncties gebruikt onderliggend de TypeScript compiler API.
Door te programmeren op de compiler die normaal gebruikt wordt om de applicaties te compileren, krijgen we toegang tot dezelfde error detectie als de compiler.
Dit zorgt ervoor dat we nieuwe bugs snel en accuraat kunnen opsporen.
Verder is de interne werking, op basis van een AST, goed gedocumenteerd.
Een basiskennis van boomstructuren is wel vereist om hiermee vlot aan de slag te gaan.

% Waarom find & replace?
Om de implementatie voor de ontwikkelaar te vereenvoudigen, wordt de TypeScript compiler API gebruikt in combinatie met zoek- en vervangfuncties op basis van regex.

\section{Angular aanpassingen}
\label{ch:angular-aanpassingen}

% Wat moet er veranderen aan de code?
Volgens de Angular update handleidling door het \textcite{AngularUpdateGuide2025} zijn er in totaal 80 verschillende stappen nodig om een applicatie van v16 naar v20 te updaten.
% Wat doen we hiermee?
Dit onderzoek verdeelt deze stappen in verschillende ``categorieën''.
% Waarom verdelen in categorieën?
Deze onderverdeling geeft een beter overzicht van wat veranderd moet worden aan een Angular v16 applicatie.
Tenslotte geeft deze onderverdeling in combinatie met de resultaten van de updater een beter inzicht in waar de updater meer geschikt voor is.
% Wat zijn de categorieën?
De categorieën zijn opgesteld als volgt:
\begin{itemize}
  \item Veranderingen aan TypeScript.
  Dit is het grootste deel van alle aanpassingen.
  \item Veranderingen aan HTML templates.
  Dit zijn aanpassingen aan Angular specifieke code in HTML templates.
  \item Veranderingen aan unittesten.
  Dit zijn aanpassingen aan de unittests die afhankelijk zijn van Angular.
  \item Veranderingen aan JSON.
  Dit zijn aanpassingen aan JSON bestanden die de applicatie configureren.
  \item Uit te voeren commando's.
  Dit zijn commando's die uitgevoerd moeten worden in de command line.
  Meestal gaat dit om Angular packages of dependencies te updaten.
  \item Veranderingen aan syntax.
  Dit zijn aanpassingen aan syntax die de werking van de applicatie niet aanpassen.
  \item Veranderingen aan semantiek.
  Dit zijn aanpassingen in de achterliggende werking van Angular.
  En/of veranderingen die ervoor zorgen dat de huidige werking van de applicatie moet veranderen.
  \item Veranderingen die niet van toepassing zijn. 
  Dit zijn aanpassingen aan functies toegevoegd na v16. 
  Het is dus onmogelijk dat de applicaties binnen Stater hiervan gebruikmaken.
\end{itemize}

% Kan een stap meerdere categorieën hebben?
Buiten de veranderingen dat niet van toepassing zijn, zijn deze categorieën niet wederzijds exclusief.
In één stap kunnen meerdere categorieën van toepassing zijn.
Een verandering kan impact hebben op zowel TypeScript als HTML, syntax als semantiek, \dots.

\section{Opzet proof of concept}
\label{ch:opzet-proof-of-concept}

\subsection{Opzet testomgeving}
\label{ch:opzet-proof-of-concept:opzet-testomgeving}

% Wat is de test opgeving?
Om de effectiviteit van de updater te meten, zet dit onderzoek een testomgeving op.
De testomgeving is een applicatie gemaakt met Angular v16.
Het bestaat uit verschillende klasse en componenten specifiek geschreven met code dat moet veranderen in de update naar v20.
De applicatie is enkel syntactisch correct, verder heeft het geen doel.
Dit wil zeggen dat het enkel moet kunnen compileren zonder fouten, meer niet.
Er is bewust gekozen geen verdere semantiek aan de testomgeving te koppelen, omdat de huidige aanpak er geen rekening mee kan houden.

% Waarvoor dient het?
Dit onderzoek kiest ervoor om de effectiviteit van de updater te testen in een gecontroleerde omgeving om een totaalbeeld te krijgen van alle stappen.
Angular is een uitgebreid framework met verschillende functionaliteiten.
In de praktijk is het niet zeker of een Angular applicatie al deze functionaliteiten gebruikt.
Neem bijvoorbeeld animaties.
Angular voorziet in functies om animaties toe te voegen aan de UI.
Niet alle applicaties hebben dit nodig.
Als er updates zijn aan deze functionaliteiten, zal de updater ze niet kunnen uitvoeren, omdat de functie in kwestie niet gebruikt wordt.
De updater testen op een willekeurige applicatie kan een verkeerd beeld schetsen van wat mogelijk is.
Het opzetten van een testomgeving geeft de mogelijkheid om een ruimer beeld te schetsen van wat mogelijk is met de updater.

% Hoe werd het opgesteld?
Het volgende proces wordt gehanteerd in het opstellen van de testapplicatie.
We doorlopen de Angular update handleidling van \textcite{AngularUpdateGuide2025} en evalueren elke stap of deze in aanmerking komt voor automatisatie.
Zoals eerder besproken zijn niet alle stappen aanpassingen aan TypeScript bestanden.
Verder is de aard van de verandering belangrijk.
De updater kan syntax interpreteren, maar geen semantiek.
Tabel~\ref{tab:opzet-testomgeving} geeft een overzicht van welke soort veranderingen in de testomgeving opgenomen worden.
\begin{table}
  \centering
  \begin{tabular}{p{0.24\textwidth}|p{0.22\textwidth}|p{0.46\textwidth}}
    \toprule
    \textbf{Soort verandering} & \textbf{In testomgeving} & \textbf{Reden} \\
    \midrule
    \multicolumn{3}{c}{\textbf{Bestandstype waar de verandering plaatsvindt}} \\
    \hline
    In HTML/JSON                        & Nee & De updater heeft geen vat op HTML/JSON. \\
    \hline
    In TypeScript                       & Ja  & De updater is gemaakt om TypeScript syntax te interpreteren. We voorzien een codefragment waarvan we weten dat het geüpdatet moet worden. \\
    \hline
    In TypeScript \& HTML/JSON          & Ja  & De updater heeft geen vat op HTML/JSON maar wel op TypeScript. We voorzien een codefragment waarvan we weten dat het geüpdatet moet worden. \\
    \hline
    \multicolumn{3}{c}{\textbf{Aard van de verandering}} \\
    \hline
    Aan semantiek                       & Nee & De updater heeft geen vat op semantiek. Dit soort aanpassingen zijn afhankelijk van hoe een applicatie Angular gebruikt. \\
    \hline
    Aan syntax                          & Ja  & De updater is gemaakt om TypeScript syntax te interpreteren. We voorzien een codefragment waarvan we weten dat het geüpdatet moet worden. \\
    \hline
    Aan syntax \& semantiek             & Ja  & De updater heeft geen vat op semantiek, maar wel op de syntax. We voorzien een codefragment waarvan we weten dat het geüpdatet moet worden. \\
    \hline
    \multicolumn{3}{c}{\textbf{Randgevalen}} \\
    \hline
    CLI-operaties                       & Nee & Voor deze stappen is geen nood aan een speciaal stuk code. \\
    \hline
    Functionaliteiten toegevoegd na v16 & Nee & Het is onmogelijk dat deze functionaliteiten gebruikt worden in de applicaties van Stater. \\
    \bottomrule
  \end{tabular}
  \caption[Opzet testomgeving]{
    \label{tab:opzet-testomgeving}
    Omschrijft welke stappen uit het updateproces al dan niet opgenomen worden in de testopgeving.
    Een stap wordt opgenomen in de testomgeving indien deze voldoet aan het bestandstype en de aard van de verandering.
  }
\end{table}
Voor elke verandering opgenomen in de testomgeving voorzien we een codefragment dat geüpdatet moet worden.
Alle codefragmenten zijn semi-realistisch en volgen de Angular best-practices waar mogelijk.
Tenslotte geven we per codefragment minstens één stuk code mee dat gelijkaardig is aan de code die veranderd moet worden.
Dit dient als controle om na te gaan of de updater specifiek genoeg ingesteld is.

\subsection{Opzet updater}
\label{ch:opzet-proof-of-concept:opzet-updater}

% Hoe wordt de updater geëvalueerd?
Dit onderzoek stelt een updater op die de testapplicatie van Angular v16 tot v20 autonoom tracht te updaten waar mogelijk.
Buiten de updates uitvoeren zal deze updater bijhouden wat het kan en niet kan.
Elke stap in het updateproces wordt manueel gecategoriseerd volgens de categorieën besproken in Hoofdstuk~\ref{ch:angular-aanpassingen}.
De updater probeert voor elk codefragment in de testapplicatie de nodige aanpassingen te detecteren en vervolgens te automatiseren.

% Woe wordt detecteer-/automatiseerbaarheid geëvalueerd?
Om een beter overzicht te krijgen van de capaciteiten van de updater worden detectie en automatisatie elk onderverdeeld in drie categorieën: niet, deels en volledig.
Niet detecteer-/automatiseerbaar zijn aanpassingen die niet autonoom uitvoerbaar zijn.
Dit kan zijn door de limitaties van de updater of de complexiteit van de aanpassing.
Dit onderzoek beschouwt een aanpassing als te complex indien er nood is om meer dan één AST te doorlopen om de aanpassing te detecteren.
Per TypeScript bestand behoort één AST.
Indien er meer dan één AST doorlopen moet worden, betekent dit dat de aanpassing betrekking heeft op meerdere bestanden binnen de applicatie.
Deels detecteer-/automatiseerbaar zijn aanpassingen die enkel deels autonoom uitvoerbaar zijn.
Bijvoorbeeld een functie die een nieuwe naam en extra parameters krijgt.
De naam kunnen we automatisch veranderen, maar de extra parameters moeten handmatig ingevuld worden, omdat ze afhankelijk zijn van de context.
Daarom kunnen uit de automatisatie van deze aanpassingen compiler errors ontstaan.
Volledig detecteer-/automatiseerbaar zijn aanpassingen die autonoom uitvoerbaar zijn.
Hier kan de aanpassing volledig autonoom gedetecteerd en/of uitgevoerd worden zonder compiler errors te veroorzaken.

% Hoe maken we de updater?
Om de updater te maken, voorziet het onderzoek verschillende helperfuncties om de implementatie te vereenvoudigen.
% Wat gebruiken we hiervoor?
Deze helperfuncties maken gebruik van de ts-morph package.
ts-morph is een open-source package ontwikkeld door \textcite{Sherret2025}.
Het is een wrapper bovenop de TypeScript Compiler API die het mogelijk maakt een TypeScript project om te zetten naar een AST.
Verder biedt het verschillende functies aan om een AST te navigeren en te manipuleren.
% Waarom nog een laag hierbovenop?
Onze helperfuncties vormen een extra laag hierbovenop.
Het doel hiervan is om een eenvoudige syntax te creëren, specifiek naar de noden van de updater.

\subsection{Validatie updater}
\label{ch:opzet-proof-of-concept:validatie-updater}

% Hoe evalueren we de updater?
Om de correctheid van de updater te testen, stelt dit onderzoek een controleomgeving op.
De controleomgeving is opgesteld door de testomgeving handmatig te updaten naar v20.
We evalueren de updater door de output te vergelijken met de controleomgeving.
Deze vergelijking gebeurt door alle TypeScript bestanden van beide projecten in te lezen en karakter per karakter te vergelijken.
De updater in het proof of concept houdt bij welke bestanden aangepast werden en welk type automatisatie van toepassing is.
Tabel~\ref{tab:evaluatie-updater} geeft een overzicht van wanneer de output van de updater correct is.
\begin{table}
  \centering
  \begin{tabular}{p{0.22\textwidth}|p{0.3\textwidth}|p{0.4\textwidth}}
    \toprule
    \textbf{Conditie} & \textbf{Correcte indien} & \textbf{Reden} \\
    \hline
    geen output            & Niet automatiseerbaar     & Er zijn geen veranderingen uitgevoerd. \\
    \hline
    output = controle      & Volledig automatiseerbaar & Er is een verandering en deze matcht de controle. \\
    \hline
    output \neq{} controle & Deels automatiseerbaar    & Er is een verandering, maar deze matcht niet volledig met de controle. \\
    \bottomrule
  \end{tabular}
  \caption[Evaluatie updater]{
    \label{tab:evaluatie-updater}
    Omschrijft wanneer de output van de updater correct is.
    De updater in het proof of concept houdt bij welke bestanden aangepast werden en welk type automatisatie van toepassing is.
  }
\end{table}



% Voeg hier je eigen hoofdstukken toe die de ``corpus'' van je bachelorproef
% vormen. De structuur en titels hangen af van je eigen onderzoek. Je kan bv.
% elke fase in je onderzoek in een apart hoofdstuk bespreken.

%\input{...}
%\input{...}
%...
%%=============================================================================
%% Proof of concept
%%=============================================================================

\chapter{Proof of concept}%
\label{ch:proof-of-concept}

% Wat gebeurt er in dit hoofdstuk?
In dit hoofdstuk geven we een technische omschrijving van hoe de updater werkt.
We beginnen met een omschrijving van de helperfuncties.
Vervolgens geven we enkele voorbeelden van hoe deze functies samen gebruikt worden om een aanpassing aan broncode te automatiseren.
\medskip

% Hoe vinden we de juiste nodes?
De eerste functie in codefragment~\ref{cf:find-nodes} helpt om de AST op een uniforme manier te navigeren.
Het doorloopt de AST vanaf een gegeven node.
Onderliggend gebruikt het een diepte-eerst-in-orde-zoekalgoritme.
De parameters van deze functie specificeren twee callbackfuncties.
De eerste is een predicaat dat nagaat of een node voldoet aan een bepaalde omschrijving.
Indien dit waar is, wordt de tweede callbackfunctie opgeroepen. 
Deze voert een bepaalde operatie uit op de node.
De functie geeft het aantal nodes terug dat aan het predicaat voldoet.
Dit maakt het mogelijk om deze functie als een predicaat mee te geven en zo de functie te nesten.
\medskip

Codefragment~\ref{cf:get-ancestor} omschrijft een andere manier om de AST te navigeren.
Hier doorlopen we de AST van een gegeven node terug naar de root van de AST.
De functie roept zichzelf recursief op tot de gewenste diepte of de root bereikt is.
\medskip

% Hoe impelmentere we find & replace?
De rest van de helperfuncties zijn predicaten voor codefragment~\ref{cf:find-nodes}.
We beginnen met codefragment~\ref{cf:contains-pattern}.
Deze functie gaat na of de tekstrepresentatie van een node een gegeven regexpatroon bevat.
De tekstrepresentatie van een node is simpelweg hoe een stuk code eruitziet in de broncode.
Bijvoorbeeld, deze functie oproepen op de root node van een AST is hetzelfde als zoeken naar een patroon in het eigenlijke bestand.
De functie oproepen op een node die een klassedeclaratie voorstelt is hetzelfde als zoeken naar een patroon in deze klasse.
Dit betekent ook dat de tekstrepresentatie van een node de tekst bevat van alle kinderen.
\medskip

\begin{listing}
  \begin{minted}{ts}
    export function findNodes(
      root: Node,
      predicate: (node: Node) => boolean | number,
      onMatch: (node: Node) => void,
    ): number {
      let matches = 0;
      root.forEachDescendant((node) => {
        if (predicate(node)) {
          matches += 1;
          onMatch(node);
        }
      });
      return matches;
    }
  \end{minted}
  \caption[Doorloop AST]
    {
      Helperfunctie die de AST vanaf een gegeven node doorloopt.
      Op basis van de callbackfuncties kunnen gericht aanpassingen uitgevoerd worden op de AST.
    }
  \label{cf:find-nodes}
\end{listing}

\begin{listing}
  \begin{minted}{ts}
    export function getAncestor(node: Node, count: number): Node | undefined {
      const parent = node.getParent();
      if (count <= 1 || !parent) return parent;
      return getAncestor(parent, --count);
    }
  \end{minted}
  \caption[Vind voorouder]{Helperfunctie die de n-de voorouder van een AST node teruggeeft.}
  \label{cf:get-ancestor}
\end{listing}

\begin{listing}
  \begin{minted}{ts}
    export function containsPattern(node: Node, pattern: string): boolean {
      const matches = node.getText().match(pattern);
      return matches !== null && matches.length > 0;
    }
  \end{minted}
  \caption[Bevat patroon]
    {
      Helperfunctie die nagaat of een patroon terug te vinden is in de tekstrepresentatie van een AST-node.
    }
  \label{cf:contains-pattern}
\end{listing}

% Hoe vinden we de diepste instantie van een patroon?
Codefragment~\ref{cf:deepest-instance-of} is een extensie op functie~\ref{cf:contains-pattern}.
Hier gaan we na of de gegeven node de diepste instantie van een patroon bevat in de AST.
Concreet controleert de functie of de huidige node, en geen enkele van de kinderen, het patroon bevat.
\medskip

% Zijn er nog abstracte functies?
Alle functies tot nu toe waren tamelijk abstract en kunnen voor meerdere doeleinden gebruikt worden.
Wat volgt zijn enkele functies die specifiek één doel hebben.
% Hoe vinden we de scope van een node?
De eerste van deze functies in codefragment~\ref{cf:in-scope-of} gaat na of een node in een bepaalde scope ligt.
We doen dit door recursief het syntaxtype van de ouder te vergelijken tot een match is gevonden of de root node bereikt is.
ts-morph evalueert het syntaxtype aan de hand van de \emph{SyntaxKind} enumeratie.
Voorbeelden van syntaxtype zijn: bestanden, importdeclaraties, klassedeclaraties, expressies, decorators, keywords, \dots.
Als een node aan het syntaxtype matcht, dan wordt deze node teruggegeven.
\medskip

% Hoe vinden we het type van de node?
Codefragment~\ref{cf:has-type} definieert een helperfunctie om het type van een node te vergelijken.
Dit doen we door de tekstrepresentatie van het type te vergelijken met een gegeven regex patroon.
De tekstrepresentatie van het type is simpelweg hoe het type gebruikt wordt in de broncode.
\medskip

% Hoe vinden we het type van het klasse dat de node aanspreekt?
De laatste helperfunctie in codefragment~\ref{cf:accessed-from} dient om na te gaan of een node toegankelijk is vanaf een bepaald type.
Deze functie kijkt of de gegeven node aangesproken wordt vanuit een klasse.
Indien dit zo is, vergelijken we het type van de klasse via de functie in codefragment~\ref{cf:has-type}.
\medskip

\begin{listing}
  \begin{minted}{ts}
    export function deepestInstanceOf(node: Node, pattern: string): boolean {
      const matchesCurrent = containsPattern(node, pattern);
      const matchesChild = node.forEachChild((child) =>
        containsPattern(child, pattern),
      );
      return matchesCurrent && !matchesChild;
    }
  \end{minted}
  \caption[Bevat diepste instantie van patroon]
    {
      Helperfunctie die nagaat of een node de diepste instantie van een patroon bevat in de AST.
    }
  \label{cf:deepest-instance-of}
\end{listing}

\begin{listing}
  \begin{minted}{ts}
    export function inScopeOf(node: Node, kind: SyntaxKind): Node | undefined {
      const parent = node.getParent();
      if (!parent) return undefined;
      if (parent.getKind() === kind) return parent;
      return inScopeOf(parent, kind);
    }
  \end{minted}
  \caption[In scope van]{Helperfunctie die nagaat of een AST node in een bepaalde scope zit.}
  \label{cf:in-scope-of}
\end{listing}

\begin{listing}
  \begin{minted}{ts}
    export function hasType(node: Node, type: string): boolean {
      const matches = node
        .getType()
        .getText(undefined, TypeFormatFlags.InTypeAlias)
        .match(type);
      return matches !== null && matches.length > 0;
    }
  \end{minted}
  \caption[Heeft type]{Helperfunctie die nagaat of een AST node een bepaald type heeft.}
  \label{cf:has-type}
\end{listing}

\begin{listing}
  \begin{minted}{ts}
    export function accessedFrom(node: Node, type: string): boolean {
      const accessProp = node.getParentIfKind(SyntaxKind.PropertyAccessExpression);
      if (!accessProp) return false;
      return hasType(accessProp.getExpression(), type);
    }
  \end{minted}
  \caption[Opgeroepen vanuit]{Helperfunctie die nagaat of een AST node opgeroepen wordt vanuit een bepaald type.}
  \label{cf:accessed-from}
\end{listing}

% Wat doen we met deze functies?
Deze functies maken het mogelijk om een AST te doorlopen in enkele lijnen code en gericht code te detecteren.
Wat volgt zijn enkele voorbeelden van hoe deze functies samen werken.
De aanpassingen die uitgevoerd worden, komen uit de Angular update handleiding door \textcite{AngularUpdateGuide2025}.
% Zijn er voorbeelden?
Het eerste voorbeeld in codefragment~\ref{cf:sample-1} toont aan hoe we de methode van een bepaalde klasse van naam veranderen.
Als predicaat zoeken we de naam van de methode op om vervolgens de naam van de klasse te vergelijken.
Indien een node aan het predicaat voldoet, vervangt het de tekst met de nieuwe naam van de methode.
\medskip

% Wat kan het nog?
De updater kan meer dan methodes van naam veranderen.
Ook alleenstaande functies zijn mogelijk.
Neem codefragment~\ref{cf:sample-2} als voorbeeld.
Hier veranderen we alle instanties van de async-functie uit Angular met waitForAsync.
TypeScript gebruikt async als sleutelwoord.
Deze instanties mogen niet mee veranderen.
Door één lijn code toe te voegen, is het mogelijk om alle instanties van het async-sleutel\-woord uit te filteren.
\medskip

% Wat als we kunnen automatiseren?
Het is niet altijd mogelijk om de nodige aanpassingen te automatiseren.
Dan nog kan het een meerwaarde zijn om deze op te sporen.
Bijvoorbeeld, in één van de stappen in het updateproces moeten Angular componenten met de OnPush-ver\-an\-der\-ing detectiestrategie nagekeken worden hoe ze interageren met templates.
We weten op voorhand dat templates niet toegankelijk zijn voor de updater.
Maar we kunnen wel verandering detectiestrategieën gaan opsporen in TypeScript.
In codefragment~\ref{cf:sample-3} zoeken we alle componenten op met de OnPush-ver\-an\-der\-ing detectiestrategie.
Vervolgens tonen we de locatie van deze component binnen het project.
Dit tonen we via de bestandsnaam, gevolgd door het lijnnummer in de code.
\medskip

% Is dit alles wat de updater kan?
Dit waren enkele voorbeelden van hoe een updater met behulp van de helperfuncties ontwikkeld kan worden.
De predicaten in deze voorbeelden zijn opzettelijk simpel gehouden.
% Werkt dit op eender welke code?
De specificiteit van een predicaat is afhankelijk van de complexiteit van het project en de manier waarop de broncode geschreven is.
Verschillende bedrijven hanteren intern verschillende manieren om code te schrijven en te structureren.
Deze voorbeelden werken perfect in de testomgeving van dit onderzoek.
Of deze even goed zullen werken in de praktijk is afhankelijk van het project.
\medskip

\begin{listing}
  \begin{minted}{ts}
    const project = loadProject();
    project.getSourceFiles().forEach((file) =>
      findNodes(
        file,
        (node) =>
          deepestInstanceOf(node, "mutate") &&
          accessedFrom(node, "WritableSignal"),
        (node) => node.replaceWithText("update"),
      ),
    );
    await saveProject(project);
  \end{minted}
  \caption[Updater voorbeeld 1]{Hernoemt alle instanties van de mutate-methode uit de WritableSignal-klasse met update.}
  \label{cf:sample-1}
\end{listing}

\begin{listing}
  \begin{minted}{ts}
    const project = loadProject();
    project.getSourceFiles().forEach((file) =>
      findNodes(
        file,
        (node) =>
          deepestInstanceOf(node, "async") &&
          node.getKind() !== SyntaxKind.AsyncKeyword,
        (node) => node.replaceWithText("waitForAsync"),
      ),
    );
    await saveProject(project);
  \end{minted}
  \caption[Updater voorbeeld 2]{Hernoem alle instanties van de async-functie met waitForAsync.}
  \label{cf:sample-2}
\end{listing}

\begin{listing}
  \begin{minted}{ts}
    const project = loadProject();
    project.getSourceFiles().forEach((file) =>
      findNodes(
        file,
        (node) =>
          deepestInstanceOf(node, "OnPush") &&
          accessedFrom(node, "ChangeDetectionStrategy") &&
          !!inScopeOf(node, SyntaxKind.Decorator),
        () => console.log(file.getBaseName(), node.getStartLineNumber()),
      ),
    );
  \end{minted}
  \caption[Updater voorbeeld 3]{Zoekt naar elke instantie van de OnPush-methode opgeroepen uit ChangeDetectionStrategy in de scope van een decorator.}
  \label{cf:sample-3}
\end{listing}

%%=============================================================================
%% Conclusie
%%=============================================================================

\chapter{Conclusie}
\label{ch:conclusie}

% TODO: Trek een duidelijke conclusie, in de vorm van een antwoord op de
% onderzoeksvra(a)g(en). Wat was jouw bijdrage aan het onderzoeksdomein en
% hoe biedt dit meerwaarde aan het vakgebied/doelgroep? 
% Reflecteer kritisch over het resultaat. In Engelse teksten wordt deze sectie
% ``Discussion'' genoemd. Had je deze uitkomst verwacht? Zijn er zaken die nog
% niet duidelijk zijn?
% Heeft het onderzoek geleid tot nieuwe vragen die uitnodigen tot verder 
%onderzoek?

\section{Test resultaten}
\label{ch:test-resultaten}

Kruistabel \ref{tab:resultaten-deel-1} en \ref{tab:resultaten-deel-2} tonen de resultaten van de updater uitgevoerd op de testomgeving.
Alle rijen buiten \emph{n.v.t.} onder \emph{Verandering} zijn overlappend, zoals besproken in hoofdstuk~\ref{ch:soorten-aanpassingen}.
De informatie in deze tabellen wordt per kolom gelezen.
Elke kolom bevat zowel een absolute als een relatieve waarde.
De relatieve waarde is berekend ten opzichte van de eerste rij in de kolom genaamd \emph{\#Stappen}.
\medskip

Uit het totaal van de 80 uit te voeren stappen blijkt dat 27,5\% volledig en 10\% gedeeltelijk automatiseerbaar is.
Dit is lager dan het verwachte resultaat van 65\% uit het onderzoeksvoorstel, zie hoofdstuk~\ref{ch:onderzoeksvoorstel}.
Eén van de factoren die een rol speelt in de automatiseerbaarheid is de aard van de aanpassing.
De updater is gelimiteerd in het opsporen van semantische aanpassingen.
In de update van v16 naar v20 waren er meer stappen met impact op semantiek dan op syntax.
Uit de stappen met impact op semantiek was amper 2,63\% automatiseerbaar.
Dit kan verklaren waarom de totale automatiseerbaarheid lager ligt dan verwacht.
De totale automatiseerbaarheid had hoger kunnen liggen indien er meer syntactische aanpassingen waren.
\medskip

Tenslotte willen we de aandacht leggen op de kolom \emph{Testen}.
Het blijkt dat 20\% van alle mogelijke stappen in deze update impact heeft op testen.
87,5\% hiervan zijn aanpassingen aan semantiek.
Dit wil zeggen dat er een verandering is in de achterliggende werking.
Testen zijn belangrijk om de werking van onze applicaties te waarborgen.
Als de update de testen aanpast, is het mogelijk dat deze niet meer betrouwbaar zijn.

% | Type                    | #Total    | %Total    | #TS       | %TS       | #Test     | %Test     | #Syntax   | %Syntax   | #Semantics   | %Semantics   | #Template   | %Template   | #JSON     | %JSON     | #CLI      | %CLI      |
% | ----------------------- | --------- | --------- | --------- | --------- | --------- | --------- | --------- | --------- | ------------ | ------------ | ----------- | ----------- | --------- | --------- | --------- | --------- |
% | #Steps                  | 80        | 100,00%   | 50        | 100,00%   | 16        | 100,00%   | 24        | 100,00%   | 38           | 100,00%      | 10          | 100,00%     | 12        | 100,00%   | 12        | 100,00%   |
% | Fully automatable       | 22        | 27,50%    | 10        | 20,00%    | 3         | 18,75%    | 9         | 37,50%    | 1            | 2,63%        | 0           | 0,00%       | 11        | 91,67%    | 12        | 100,00%   |
% | Partially automatable   | 8         | 10,00%    | 8         | 16,00%    | 0         | 0,00%     | 5         | 20,83%    | 5            | 13,16%       | 0           | 0,00%       | 0         | 0,00%     | 0         | 0,00%     |
% | Not automatable         | 50        | 62,50%    | 32        | 64,00%    | 13        | 81,25%    | 10        | 41,67%    | 32           | 84,21%       | 10          | 100,00%     | 1         | 8,33%     | 0         | 0,00%     |
% | Fully detectable        | 24        | 30,00%    | 24        | 48,00%    | 6         | 37,50%    | 12        | 50,00%    | 13           | 34,21%       | 0           | 0,00%       | 0         | 0,00%     | 0         | 0,00%     |
% | Partially detectable    | 5         | 6,25%     | 5         | 10,00%    | 0         | 0,00%     | 2         | 8,33%     | 4            | 10,53%       | 1           | 10,00%      | 0         | 0,00%     | 0         | 0,00%     |
% | Not detectable          | 51        | 63,75%    | 21        | 42,00%    | 10        | 62,50%    | 10        | 41,67%    | 21           | 55,26%       | 9           | 90,00%      | 12        | 100,00%   | 12        | 100,00%   |
% | Change in TypeScript    | 50        | 62,50%    | 50        | 100,00%   | 16        | 100,00%   | 17        | 70,83%    | 37           | 97,37%       | 3           | 30,00%      | 0         | 0,00%     | 0         | 0,00%     |
% | Change in template      | 10        | 12,50%    | 3         | 6,00%     | 2         | 12,50%    | 6         | 25,00%    | 4            | 10,53%       | 10          | 100,00%     | 0         | 0,00%     | 0         | 0,00%     |
% | Change in test          | 16        | 20,00%    | 16        | 32,00%    | 16        | 100,00%   | 2         | 8,33%     | 14           | 36,84%       | 2           | 20,00%      | 0         | 0,00%     | 0         | 0,00%     |
% | Change in JSON          | 12        | 15,00%    | 0         | 0,00%     | 0         | 0,00%     | 1         | 4,17%     | 0            | 0,00%        | 0           | 0,00%       | 12        | 100,00%   | 11        | 91,67%    |
% | Change in CLI           | 12        | 15,00%    | 0         | 0,00%     | 0         | 0,00%     | 0         | 0,00%     | 0            | 0,00%        | 0           | 0,00%       | 11        | 91,67%    | 12        | 100,00%   |
% | Change not applicable   | 10        | 12,50%    | 0         | 0,00%     | 0         | 0,00%     | 0         | 0,00%     | 0            | 0,00%        | 0           | 0,00%       | 0         | 0,00%     | 0         | 0,00%     |
% | Change to syntax        | 24        | 30,00%    | 17        | 34,00%    | 2         | 12,50%    | 24        | 100,00%   | 4            | 10,53%       | 6           | 60,00%      | 1         | 8,33%     | 0         | 0,00%     |
% | Change to semantics     | 38        | 47,50%    | 37        | 74,00%    | 14        | 87,50%    | 4         | 16,67%    | 38           | 100,00%      | 4           | 40,00%      | 0         | 0,00%     | 0         | 0,00%     |

\begin{table}
  \centering
  \begin{tabular}{l*{5}{|lr}}
    \toprule
    \textbf{Categorie} & \multicolumn{2}{c|}{\textbf{Totaal}} & \multicolumn{2}{c|}{\textbf{TypeScript}} & \multicolumn{2}{c|}{\textbf{Testen}} & \multicolumn{2}{c|}{\textbf{Syntax}} & \multicolumn{2}{c}{\textbf{Semantiek}} \\
    \hline
    \textbf{\#Stappen}  & 80 & 100,00\%  & 50 & 100,00\%  & 16 & 100,00\% & 24 & 100,00\%  & 38 & 100,00\% \\
    \hline                                                      
    \multicolumn{11}{c}{\textbf{Automatiseerbaar}} \\                     
    \hline                                                      
    \textbf{Volledig}   & 22 &  27,50\%  & 10 &  20,00\%  & 3  &  18,75\% & 9  &  37,50\%  & 1  &   2,63\% \\
    \textbf{Gedeeltelijk}      & 8  &  10,00\%  & 8  &  16,00\%  & 0  &   0,00\% & 5  &  20,83\%  & 5  &  13,16\% \\
    \textbf{Niet}       & 50 &  62,50\%  & 32 &  64,00\%  & 13 &  81,25\% & 10 &  41,67\%  & 32 &  84,21\% \\
    \hline                                                      
    \multicolumn{11}{c}{\textbf{Detecteerbaar}} \\                        
    \hline                                                       
    \textbf{Volledig}   & 24 &   30,00\%  & 24 &  48,00\%  & 6  &  37,50\% & 12 &  50,00\%  & 13 &  34,21\% \\
    \textbf{Gedeeltelijk}      & 5  &    6,25\%  & 5  &  10,00\%  & 0  &   0,00\% & 2  &   8,33\%  & 4  &  10,53\% \\
    \textbf{Niet}       & 51 &   63,75\%  & 21 &  42,00\%  & 10 &  62,50\% & 10 &  41,67\%  & 21 &  55,26\% \\
    \hline                                                      
    \multicolumn{11}{c}{\textbf{Verandering}} \\                          
    \hline                                                      
    \textbf{TypeScript} & 50 &  62,50\%  & 50 & 100,00\%  & 16 & 100,00\% & 17 &  70,83\%  & 37 &  97,37\% \\
    \textbf{Template}   & 10 &  12,50\%  & 3  &   6,00\%  & 2  &  12,50\% & 6  &  25,00\%  & 4  &  10,53\% \\
    \textbf{Testen}     & 16 &  20,00\%  & 16 &  32,00\%  & 16 & 100,00\% & 2  &   8,33\%  & 14 &  36,84\% \\
    \textbf{JSON}       & 12 &  15,00\%  & 0  &   0,00\%  & 0  &   0,00\% & 1  &   4,17\%  & 0  &   0,00\% \\
    \textbf{CLI}        & 12 &  15,00\%  & 0  &   0,00\%  & 0  &   0,00\% & 0  &   0,00\%  & 0  &   0,00\% \\
    \textbf{n.v.t.}     & 10 &  12,50\%  & 0  &   0,00\%  & 0  &   0,00\% & 0  &   0,00\%  & 0  &   0,00\% \\
    \hline                                                      
    \textbf{Syntax}     & 24 &  30,00\%  & 17 &  34,00\%  & 2  &  12,50\% & 24 & 100,00\%  & 4  &  10,53\% \\
    \textbf{Semantiek}  & 38 &  47,50\%  & 37 &  74,00\%  & 14 &  87,50\% & 4  &  16,67\%  & 38 & 100,00\% \\
    \bottomrule
  \end{tabular}
  \caption[Resultaten deel 1]{
    \label{tab:resultaten-deel-1}Deel 1 van de resultaten van de updater uitgevoerd op de testomgeving.
    Alle rijen buiten \emph{n.v.t.} onder \emph{Verandering} zijn overlappend, zoals besproken in hoofdstuk~\ref{ch:soorten-aanpassingen}.
  }
\end{table}

\begin{table}
  \centering
  \begin{tabular}{l*{4}{|lr}}
    \toprule
    \textbf{Categorie} & \multicolumn{2}{c|}{\textbf{Totaal}} & \multicolumn{2}{c|}{\textbf{Templates}} & \multicolumn{2}{c|}{\textbf{JSON}} & \multicolumn{2}{c}{\textbf{CLI}} \\
    \hline
    \textbf{\#Stappen}   & 80 & 100,00\% & 10 & 100,00\% & 12 & 100,00\% & 12 & 100,00\% \\
    \hline                                                                     
    \multicolumn{9}{c}{\textbf{Automatiseerbaar}} \\                                    
    \hline                                                                     
    \textbf{Volledig}    & 22 &  27,50\% & 0  &   0,00\% & 11 &  91,67\% & 12 & 100,00\% \\
    \textbf{Gedeeltelijk}       & 8  &  10,00\% & 0  &   0,00\% & 0  &   0,00\% & 0  &   0,00\% \\
    \textbf{Niet}        & 50 &  62,50\% & 10 & 100,00\% & 1  &   8,33\% & 0  &   0,00\% \\
    \hline                                                                     
    \multicolumn{9}{c}{\textbf{Detecteerbaar}} \\                                       
    \hline                                                                     
    \textbf{Volledig}    & 24 &  30,00\% & 0  &   0,00\% & 0  &   0,00\% & 0  &   0,00\% \\
    \textbf{Gedeeltelijk}       & 5  &   6,25\% & 1  &  10,00\% & 0  &   0,00\% & 0  &   0,00\% \\
    \textbf{Niet}        & 51 &  63,75\% & 9  &  90,00\% & 12 & 100,00\% & 12 & 100,00\% \\
    \hline                                                                     
    \multicolumn{9}{c}{\textbf{Verandering}} \\                                         
    \hline                                                                     
    \textbf{TypeScript}  & 50 &  62,50\% & 3  &  30,00\% & 0  &   0,00\% & 0  &   0,00\% \\
    \textbf{Templates}   & 10 &  12,50\% & 10 & 100,00\% & 0  &   0,00\% & 0  &   0,00\% \\
    \textbf{Testen}      & 16 &  20,00\% & 2  &  20,00\% & 0  &   0,00\% & 0  &   0,00\% \\
    \textbf{JSON}        & 12 &  15,00\% & 0  &   0,00\% & 12 & 100,00\% & 11 &  91,67\% \\
    \textbf{CLI}         & 12 &  15,00\% & 0  &   0,00\% & 11 &  91,67\% & 12 & 100,00\% \\
    \textbf{n.v.t.}      & 10 &  12,50\% & 0  &   0,00\% & 0  &   0,00\% & 0  &   0,00\% \\
    \hline                                                                     
    \textbf{Syntax}      & 24 &  30,00\% & 6  &  60,00\% & 1  &   8,33\% & 0  &   0,00\% \\
    \textbf{Semantiek}   & 38 &  47,50\% & 4  &  40,00\% & 0  &   0,00\% & 0  &   0,00\% \\
    \bottomrule
  \end{tabular}
  \caption[Resultaten deel 2]{
    \label{tab:resultaten-deel-2}Deel 2 van de resultaten van de updater uitgevoerd op de testomgeving.
    Alle rijen buiten \emph{n.v.t.} onder \emph{Verandering} zijn overlappend, zoals besproken in hoofdstuk~\ref{ch:soorten-aanpassingen}.
  }
\end{table}

\clearpage
\section{Besluit}
\label{ch:besluit}

De testresultaten tonen aan dat de updater een meerwaarde biedt in de ondersteuning van het updateproces.
Ondanks dat het verwachte resultaat niet bereikt is, was het nog steeds mogelijk om een vierde van alle aanpassingen automatisch uit te voeren.
De hoge flexibiliteit van onze aanpak maakt het mogelijk om de updater te herconfigureren voor toekomstige updates.
Bovendien is de updater niet gelimiteerd aan het uitvoeren van Angular-updates.
Dezelfde manier van werken kan toegepast worden om meer algemene refactoringen uit te voeren.
Verder kan men de updater configureren om op andere TypeScript-applicaties te werken.
\medskip

Onze aanpak is gericht op het updaten van meerdere enterprise-applicaties.
We zien in dat dit onderzoek niet in alle gevallen een meerwaarde biedt.
De reële tijdswinst van deze aanpak is afhankelijk van hoe de updater gebruikt wordt en door wie.
De updater configureren om één lijn code aan te passen in één enkele applicatie is contraproductief.
Om het meeste uit de updater te halen, moet de tijd voor de updater te configureren kleiner zijn dan de tijd om de update handmatig uit te voeren.
Deze rekensom is afhankelijk van verschillende variabelen.
Voornamelijk de kennis en ervaring van de persoon die de updater configureert.
Iemand met een diepe kennis over de code van het bedrijf zal deze som beter kunnen inschatten dan iemand zonder deze kennis.
\medskip

In totaal waren er 80 verschillende soorten aanpassingen nodig om een Angular-applicatie van v16 naar v20 te updaten.
Zoals omschreven in hoofdstuk~\ref{ch:stand-van-zaken:angular:aanpassingen-tussen-v16-en-v20} \& \ref{ch:soorten-aanpassingen} hadden deze aanpassingen betrekking op verschillende aspecten van het framework.
Verder hebben we een onderscheid kunnen maken tussen het soort aanpassingen, syntactisch of semantisch.
\medskip

Er bestaan meerdere manieren om code automatisch aan te passen.
Dit kan via tools ingebouwd in IDE's, algoritmes om fouten op te sporen of aan te passen, en machine- of deep learning-modellen.
Voor elke manier zijn er verschillende implementaties beschikbaar.
In dit onderzoek legden we de nadruk op de meest voorkomende: zoek- \& vervangfuncties, language-servers en compiler-tooling.
\medskip

Zoek- \& vervangfuncties in combinatie met compiler-tooling zijn gekozen als de meest geschikte manier om in deze casus toe te passen.
Dit onderzoek heeft hiervoor gekozen vanwege de hoge kans op een succesvolle implementatie en de betrouwbaarheid van de output.
Zoek- \& vervangfuncties zijn welbekend en simpel om mee te werken en te implementeren.
Om hun tekortkomingen te compenseren, werd compiler-tooling gebruikt via de TypeScript Compiler API.
Meerdere studies tonen aan dat compiler-tooling werkt op grote schaal.
Verder geeft het ons dezelfde errordetectie van de compiler, waardoor we op een betrouwbare manier bugs kunnen opsporen.
\medskip

In hoofdstuk~\ref{ch:stand-van-zaken:refactoring:known-problems} bespraken we kort wat de gekende problemen waren bij het refactoren van code.
In de context van onze aanpak kunnen we zeggen dat aanpassingen aan semantiek problematisch zijn om te refactoren.
Er is zowel kennis nodig van de werking van Angular als van de applicaties die Angular gebruiken.
In de update van v16 naar v20 blijkt dat 47,50\% van alle aanpassingen betrekking heeft op de achterliggende semantiek van Angular.
Zelfs als een aanpassing betrekking heeft op zowel syntax als semantiek, blijkt dit moeilijk te automatiseren.

\section{Verder onderzoek}
\label{ch:verder-onderzoek}

In de stand van zaken hebben we verschillende automatisatietechnieken besproken.
Een vergelijking van deze technieken voor toepassing in andere casussen kan waardevol zijn.
Wanneer zou een integratie met een language server, of met AI, gepast zijn bijvoorbeeld?
\medskip

Tijdens het schrijven van dit onderzoek is Angular v21 uitgekomen.
Deze versie komt met nieuwe tools om AI beter te integreren in het ontwikkelingsproces.
Voornamelijk beweert het \textcite{AngularV21Announcement2025} dat het AI toelaat om de nieuwste functionaliteiten te gebruiken.
Dit was één van de redenen dat AI niet gekozen werd in dit onderzoek.
Deze tools zijn momenteel nog experimenteel, maar kunnen veelbelovend zijn.



%---------- Bijlagen -----------------------------------------------------------

\appendix

\chapter{Onderzoeksvoorstel}
\label{ch:onderzoeksvoorstel}

Het onderwerp van deze bachelorproef is gebaseerd op een onderzoeksvoorstel dat vooraf werd beoordeeld door de promotor. Dat voorstel is opgenomen in deze bijlage.

\section*{Samenvatting}

% Kopieer en plak hier de samenvatting (abstract) van je onderzoeksvoorstel.

% Wat is het thema?
Een applicatie updaten naar een nieuwe versie van een softwareframework kan veel tijd in beslag nemen, zeker als de applicatie meerdere versies achterloopt.
Hetzelfde updateproces vervolgens herhalen over meerdere applicaties zorgt ervoor dat de onderhoudstijd snel toeneemt.
% Wat is de onderzoeksvraag?
Het doel van dit onderzoek is om de update van het Angular-webframework van versie 16 naar versie 20 in meerdere applicaties te automatiseren, met als doel de onderhoudstijd voor de ontwikkelaars te verlagen.
% Wat is het probleem? 
Framework updates kunnen de performantie en veiligheid van de applicatie verbeteren.
Het is daarom belangrijk om deze tijdig uit te voeren.
Deze updates zijn niet altijd even simpel om toe te passen, zeker als deze gepaard gaan met veranderingen aan de broncode van de applicatie.
In grotere broncodes neemt het proces om alle veranderingen systematisch aan te brengen veel tijd in beslag.
% Wat gaat er gebeuren in dit onderzoek?
Dit onderzoek start met het maken van een oplijsting van alle nodige aanpassingen tussen Angular versie 16 en versie 20.
Vervolgens worden de verschillende manieren om aanpassingen automatisch uit te voeren onderzocht.
% Hoe gaat het onderzoek uitgevoerd worden?
Op basis hiervan wordt een proof of concept ontwikkeld die het updateproces zal automatiseren waar mogelijk.
Deze proof of concept wordt getest in een gecontroleerde omgeving.
% Wat is denk je dat het resultaat is van het onderzoek?
De effectiviteit wordt beoordeeld aan de hand van het aantal nodige veranderingen die het automatisatieproces correct kan opsporen en uitvoeren ten opzichte van het totaal aantal uit te voeren aanpassingen.
Dit onderzoek verwacht dat het 65\% van alle nodige veranderingen kan automatiseren.
% Wat hebben we aan het resultaat?
Het automatisatieproces zal naar verwachting de nodige tijd om de applicaties te updaten verminderen.

% Verwijzing naar het bestand met de inhoud van het onderzoeksvoorstel

\section{Inleiding}
\label{sec:inleiding}

Het bedrijf Stater maakt momenteel gebruik van het web application framework Angular voor het maken van verschillende applicaties.
De huidige versie van Angular die binnen Stater in gebruik is, is Angular versie 16.
Op dit moment is de laatste stabiele versie van Angular, versie 20.

Stater wil graag de huidige applicaties updaten naar versie 20.
\textcite{Angular} voorziet tooling voor een applicatie automatisch naar een nieuwe versie te updaten.
Maar volgens dezeflde bron is dit gelimiteerd tot applicaties dat enkel 1 versie uiteenlopen.
Dit betekend dat de update handmatig zal uitgevoerd moeten worden.
Gezien de huidige schaal van de codebase zal dit veel tijd in beslag nemen.
Hieruit komt de vraag: Is het mogelijk om een web applicatie in Angular versie 16 automatisch te updaten naar Angular versie 20.

Voor deze vraag te beantwoorden worden volgende deelvragen geformuleerd:

\begin{itemize}
  \item Wat zijn de veranderingen tussen Angular versie 16 en Angular versie 20?
  \item Welke van deze veranderingen kunnen automatisch toegepast worden zonder de functionele of niet-functionele vereisten van de applicatie in drang te brengen?
  \item Welke manieren besteen er om code automatisch aan te passen?
  \item Wat is de meest geschikte manier om code automatisch aan te passen voor de applicaties van dit bedrijf?
\end{itemize}

Het automatisatie process zal uitgevoerd worden op een Angular applicatie waarvan op voorhand geweten is hoeveel aanpassingen er moeten gebeuren.
De effectiviteid het het process wordt beoordeeld aan de hand van het aantal aanpassingen automatisch uitgevoerd kunnen worden ten opzichte van het totaal aantal aanpassingen.

In de volgende sectie wordt een overzicht gegeven van de stand van zaken binnen het probleem en onderzoeksdomein.
Vervolgens wordt de methodologie van dit onderzoek beschreven.
En tenslotte worden de verwachte resultaten besproken, waarin de bevindignen worden samengevat.

\section{Literatuurstudie}
\label{sec:literatuurstudie}

Hier beschrijf je de \emph{state-of-the-art} rondom je gekozen onderzoeksdomein, d.w.z.\ een inleidende, doorlopende tekst over het onderzoeksdomein van je bachelorproef. Je steunt daarbij heel sterk op de professionele \emph{vakliteratuur}, en niet zozeer op populariserende teksten voor een breed publiek. Wat is de huidige stand van zaken in dit domein, en wat zijn nog eventuele open vragen (die misschien de aanleiding waren tot je onderzoeksvraag!)?

Je mag de titel van deze sectie ook aanpassen (literatuurstudie, stand van zaken, enz.). Zijn er al gelijkaardige onderzoeken gevoerd? Wat concluderen ze? Wat is het verschil met jouw onderzoek?

Verwijs bij elke introductie van een term of bewering over het domein naar de vakliteratuur, bijvoorbeeld~\autocite{Hykes2013}! Denk zeker goed na welke werken je refereert en waarom.

Draag zorg voor correcte literatuurverwijzingen! Een bronvermelding hoort thuis \emph{binnen} de zin waar je je op die bron baseert, dus niet er buiten! Maak meteen een verwijzing als je gebruik maakt van een bron. Doe dit dus \emph{niet} aan het einde van een lange paragraaf. Baseer nooit teveel aansluitende tekst op eenzelfde bron.

Als je informatie over bronnen verzamelt in JabRef, zorg er dan voor dat alle nodige info aanwezig is om de bron terug te vinden (zoals uitvoerig besproken in de lessen Research Methods).

% Voor literatuurverwijzingen zijn er twee belangrijke commando's:
% \autocite{KEY} => (Auteur, jaartal) Gebruik dit als de naam van de auteur
%   geen onderdeel is van de zin.
% \textcite{KEY} => Auteur (jaartal)  Gebruik dit als de auteursnaam wel een
%   functie heeft in de zin (bv. ``Uit onderzoek door Doll & Hill (1954) bleek
%   ...'')

Je mag deze sectie nog verder onderverdelen in subsecties als dit de structuur van de tekst kan verduidelijken.

%---------- Methodologie ------------------------------------------------------
\section{Methodologie}
\label{sec:methodologie}

Hier beschrijf je hoe je van plan bent het onderzoek te voeren. Welke onderzoekstechniek ga je toepassen om elk van je onderzoeksvragen te beantwoorden? Gebruik je hiervoor literatuurstudie, interviews met belanghebbenden (bv.~voor requirements-analyse), experimenten, simulaties, vergelijkende studie, risico-analyse, PoC, \ldots?

Valt je onderwerp onder één van de typische soorten bachelorproeven die besproken zijn in de lessen Research Methods (bv.\ vergelijkende studie of risico-analyse)? Zorg er dan ook voor dat we duidelijk de verschillende stappen terug vinden die we verwachten in dit soort onderzoek!

Vermijd onderzoekstechnieken die geen objectieve, meetbare resultaten kunnen opleveren. Enquêtes, bijvoorbeeld, zijn voor een bachelorproef informatica meestal \textbf{niet geschikt}. De antwoorden zijn eerder meningen dan feiten en in de praktijk blijkt het ook bijzonder moeilijk om voldoende respondenten te vinden. Studenten die een enquête willen voeren, hebben meestal ook geen goede definitie van de populatie, waardoor ook niet kan aangetoond worden dat eventuele resultaten representatief zijn.

Uit dit onderdeel moet duidelijk naar voor komen dat je bachelorproef ook technisch voldoen\-de diepgang zal bevatten. Het zou niet kloppen als een bachelorproef informatica ook door bv.\ een student marketing zou kunnen uitgevoerd worden.

Je beschrijft ook al welke tools (hardware, software, diensten, \ldots) je denkt hiervoor te gebruiken of te ontwikkelen.

Probeer ook een tijdschatting te maken. Hoe lang zal je met elke fase van je onderzoek bezig zijn en wat zijn de concrete \emph{deliverables} in elke fase?

%---------- Verwachte resultaten ----------------------------------------------
\section{Verwacht resultaat, conclusie}
\label{sec:verwachte_resultaten}

Hier beschrijf je welke resultaten je verwacht. Als je metingen en simulaties uitvoert, kan je hier al mock-ups maken van de grafieken samen met de verwachte conclusies. Benoem zeker al je assen en de onderdelen van de grafiek die je gaat gebruiken. Dit zorgt ervoor dat je concreet weet welk soort data je moet verzamelen en hoe je die moet meten.

Wat heeft de doelgroep van je onderzoek aan het resultaat? Op welke manier zorgt jouw bachelorproef voor een meerwaarde?

Hier beschrijf je wat je verwacht uit je onderzoek, met de motivatie waarom. Het is \textbf{niet} erg indien uit je onderzoek andere resultaten en conclusies vloeien dan dat je hier beschrijft: het is dan juist interessant om te onderzoeken waarom jouw hypothesen niet overeenkomen met de resultaten.



%%---------- Andere bijlagen --------------------------------------------------
% TODO: Voeg hier eventuele andere bijlagen toe. Bv. als je deze BP voor de
% tweede keer indient, een overzicht van de verbeteringen t.o.v. het origineel.
%\input{...}

%%---------- Backmatter, referentielijst ---------------------------------------

\backmatter{}

\setlength\bibitemsep{2pt} %% Add Some space between the bibliograpy entries
\printbibliography[heading=bibintoc]

\end{document}
