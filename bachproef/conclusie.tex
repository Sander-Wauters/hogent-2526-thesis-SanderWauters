%%=============================================================================
%% Conclusie
%%=============================================================================

\chapter{Conclusie}
\label{ch:conclusie}

% TODO: Trek een duidelijke conclusie, in de vorm van een antwoord op de
% onderzoeksvra(a)g(en). Wat was jouw bijdrage aan het onderzoeksdomein en
% hoe biedt dit meerwaarde aan het vakgebied/doelgroep? 
% Reflecteer kritisch over het resultaat. In Engelse teksten wordt deze sectie
% ``Discussion'' genoemd. Had je deze uitkomst verwacht? Zijn er zaken die nog
% niet duidelijk zijn?
% Heeft het onderzoek geleid tot nieuwe vragen die uitnodigen tot verder 
%onderzoek?

\section{Test resultaten}
\label{ch:test-resultaten}

Kruistabel \ref{tab:resultaten-deel-1} en \ref{tab:resultaten-deel-2} tonen de resultaten van de updater uitgevoerd op de testomgeving.
Alle rijen buiten \emph{n.v.t.} onder \emph{Verandering} zijn overlappend, zoals besproken in hoofdstuk~\ref{ch:soorten-aanpassingen}.
De informatie in deze tabellen wordt per kolom gelezen.
Elke kolom bevat zowel een absolute als een relatieve waarde.
De relatieve waarde is berekend ten opzichte van de eerste rij in de kolom genaamd \emph{\#Stappen}.
\medskip

Uit het totaal van de 80 uit te voeren stappen blijkt dat 27,5\% volledig en 10\% gedeeltelijk automatiseerbaar is.
Dit is lager dan het verwachte resultaat van 65\% uit het onderzoeksvoorstel, zie hoofdstuk~\ref{ch:onderzoeksvoorstel}.
Eén van de factoren die een rol speelt in de automatiseerbaarheid is de aard van de aanpassing.
De updater is gelimiteerd in het opsporen van semantische aanpassingen.
In de update van v16 naar v20 waren er meer stappen met impact op semantiek dan op syntax.
Uit de stappen met impact op semantiek was amper 2,63\% automatiseerbaar.
Dit kan verklaren waarom de totale automatiseerbaarheid lager ligt dan verwacht.
De totale automatiseerbaarheid had hoger kunnen liggen indien er meer syntactische aanpassingen waren.
\medskip

Tenslotte willen we de aandacht leggen op de kolom \emph{Testen}.
Het blijkt dat 20\% van alle mogelijke stappen in deze update impact heeft op testen.
87,5\% hiervan zijn aanpassingen aan semantiek.
Dit wil zeggen dat er een verandering is in de achterliggende werking.
Testen zijn belangrijk om de werking van onze applicaties te waarborgen.
Als de update de testen aanpast, is het mogelijk dat deze niet meer betrouwbaar zijn.

% | Type                    | #Total    | %Total    | #TS       | %TS       | #Test     | %Test     | #Syntax   | %Syntax   | #Semantics   | %Semantics   | #Template   | %Template   | #JSON     | %JSON     | #CLI      | %CLI      |
% | ----------------------- | --------- | --------- | --------- | --------- | --------- | --------- | --------- | --------- | ------------ | ------------ | ----------- | ----------- | --------- | --------- | --------- | --------- |
% | #Steps                  | 80        | 100,00%   | 50        | 100,00%   | 16        | 100,00%   | 24        | 100,00%   | 38           | 100,00%      | 10          | 100,00%     | 12        | 100,00%   | 12        | 100,00%   |
% | Fully automatable       | 22        | 27,50%    | 10        | 20,00%    | 3         | 18,75%    | 9         | 37,50%    | 1            | 2,63%        | 0           | 0,00%       | 11        | 91,67%    | 12        | 100,00%   |
% | Partially automatable   | 8         | 10,00%    | 8         | 16,00%    | 0         | 0,00%     | 5         | 20,83%    | 5            | 13,16%       | 0           | 0,00%       | 0         | 0,00%     | 0         | 0,00%     |
% | Not automatable         | 50        | 62,50%    | 32        | 64,00%    | 13        | 81,25%    | 10        | 41,67%    | 32           | 84,21%       | 10          | 100,00%     | 1         | 8,33%     | 0         | 0,00%     |
% | Fully detectable        | 24        | 30,00%    | 24        | 48,00%    | 6         | 37,50%    | 12        | 50,00%    | 13           | 34,21%       | 0           | 0,00%       | 0         | 0,00%     | 0         | 0,00%     |
% | Partially detectable    | 5         | 6,25%     | 5         | 10,00%    | 0         | 0,00%     | 2         | 8,33%     | 4            | 10,53%       | 1           | 10,00%      | 0         | 0,00%     | 0         | 0,00%     |
% | Not detectable          | 51        | 63,75%    | 21        | 42,00%    | 10        | 62,50%    | 10        | 41,67%    | 21           | 55,26%       | 9           | 90,00%      | 12        | 100,00%   | 12        | 100,00%   |
% | Change in TypeScript    | 50        | 62,50%    | 50        | 100,00%   | 16        | 100,00%   | 17        | 70,83%    | 37           | 97,37%       | 3           | 30,00%      | 0         | 0,00%     | 0         | 0,00%     |
% | Change in template      | 10        | 12,50%    | 3         | 6,00%     | 2         | 12,50%    | 6         | 25,00%    | 4            | 10,53%       | 10          | 100,00%     | 0         | 0,00%     | 0         | 0,00%     |
% | Change in test          | 16        | 20,00%    | 16        | 32,00%    | 16        | 100,00%   | 2         | 8,33%     | 14           | 36,84%       | 2           | 20,00%      | 0         | 0,00%     | 0         | 0,00%     |
% | Change in JSON          | 12        | 15,00%    | 0         | 0,00%     | 0         | 0,00%     | 1         | 4,17%     | 0            | 0,00%        | 0           | 0,00%       | 12        | 100,00%   | 11        | 91,67%    |
% | Change in CLI           | 12        | 15,00%    | 0         | 0,00%     | 0         | 0,00%     | 0         | 0,00%     | 0            | 0,00%        | 0           | 0,00%       | 11        | 91,67%    | 12        | 100,00%   |
% | Change not applicable   | 10        | 12,50%    | 0         | 0,00%     | 0         | 0,00%     | 0         | 0,00%     | 0            | 0,00%        | 0           | 0,00%       | 0         | 0,00%     | 0         | 0,00%     |
% | Change to syntax        | 24        | 30,00%    | 17        | 34,00%    | 2         | 12,50%    | 24        | 100,00%   | 4            | 10,53%       | 6           | 60,00%      | 1         | 8,33%     | 0         | 0,00%     |
% | Change to semantics     | 38        | 47,50%    | 37        | 74,00%    | 14        | 87,50%    | 4         | 16,67%    | 38           | 100,00%      | 4           | 40,00%      | 0         | 0,00%     | 0         | 0,00%     |

\begin{table}
  \centering
  \begin{tabular}{l*{5}{|lr}}
    \toprule
    \textbf{Categorie} & \multicolumn{2}{c|}{\textbf{Totaal}} & \multicolumn{2}{c|}{\textbf{TypeScript}} & \multicolumn{2}{c|}{\textbf{Testen}} & \multicolumn{2}{c|}{\textbf{Syntax}} & \multicolumn{2}{c}{\textbf{Semantiek}} \\
    \hline
    \textbf{\#Stappen}  & 80 & 100,00\%  & 50 & 100,00\%  & 16 & 100,00\% & 24 & 100,00\%  & 38 & 100,00\% \\
    \hline                                                      
    \multicolumn{11}{c}{\textbf{Automatiseerbaar}} \\                     
    \hline                                                      
    \textbf{Volledig}   & 22 &  27,50\%  & 10 &  20,00\%  & 3  &  18,75\% & 9  &  37,50\%  & 1  &   2,63\% \\
    \textbf{Gedeeltelijk}      & 8  &  10,00\%  & 8  &  16,00\%  & 0  &   0,00\% & 5  &  20,83\%  & 5  &  13,16\% \\
    \textbf{Niet}       & 50 &  62,50\%  & 32 &  64,00\%  & 13 &  81,25\% & 10 &  41,67\%  & 32 &  84,21\% \\
    \hline                                                      
    \multicolumn{11}{c}{\textbf{Detecteerbaar}} \\                        
    \hline                                                       
    \textbf{Volledig}   & 24 &   30,00\%  & 24 &  48,00\%  & 6  &  37,50\% & 12 &  50,00\%  & 13 &  34,21\% \\
    \textbf{Gedeeltelijk}      & 5  &    6,25\%  & 5  &  10,00\%  & 0  &   0,00\% & 2  &   8,33\%  & 4  &  10,53\% \\
    \textbf{Niet}       & 51 &   63,75\%  & 21 &  42,00\%  & 10 &  62,50\% & 10 &  41,67\%  & 21 &  55,26\% \\
    \hline                                                      
    \multicolumn{11}{c}{\textbf{Verandering}} \\                          
    \hline                                                      
    \textbf{TypeScript} & 50 &  62,50\%  & 50 & 100,00\%  & 16 & 100,00\% & 17 &  70,83\%  & 37 &  97,37\% \\
    \textbf{Template}   & 10 &  12,50\%  & 3  &   6,00\%  & 2  &  12,50\% & 6  &  25,00\%  & 4  &  10,53\% \\
    \textbf{Testen}     & 16 &  20,00\%  & 16 &  32,00\%  & 16 & 100,00\% & 2  &   8,33\%  & 14 &  36,84\% \\
    \textbf{JSON}       & 12 &  15,00\%  & 0  &   0,00\%  & 0  &   0,00\% & 1  &   4,17\%  & 0  &   0,00\% \\
    \textbf{CLI}        & 12 &  15,00\%  & 0  &   0,00\%  & 0  &   0,00\% & 0  &   0,00\%  & 0  &   0,00\% \\
    \textbf{n.v.t.}     & 10 &  12,50\%  & 0  &   0,00\%  & 0  &   0,00\% & 0  &   0,00\%  & 0  &   0,00\% \\
    \hline                                                      
    \textbf{Syntax}     & 24 &  30,00\%  & 17 &  34,00\%  & 2  &  12,50\% & 24 & 100,00\%  & 4  &  10,53\% \\
    \textbf{Semantiek}  & 38 &  47,50\%  & 37 &  74,00\%  & 14 &  87,50\% & 4  &  16,67\%  & 38 & 100,00\% \\
    \bottomrule
  \end{tabular}
  \caption[Resultaten deel 1]{
    \label{tab:resultaten-deel-1}Deel 1 van de resultaten van de updater uitgevoerd op de testomgeving.
    Alle rijen buiten \emph{n.v.t.} onder \emph{Verandering} zijn overlappend, zoals besproken in hoofdstuk~\ref{ch:soorten-aanpassingen}.
  }
\end{table}

\begin{table}
  \centering
  \begin{tabular}{l*{4}{|lr}}
    \toprule
    \textbf{Categorie} & \multicolumn{2}{c|}{\textbf{Totaal}} & \multicolumn{2}{c|}{\textbf{Templates}} & \multicolumn{2}{c|}{\textbf{JSON}} & \multicolumn{2}{c}{\textbf{CLI}} \\
    \hline
    \textbf{\#Stappen}   & 80 & 100,00\% & 10 & 100,00\% & 12 & 100,00\% & 12 & 100,00\% \\
    \hline                                                                     
    \multicolumn{9}{c}{\textbf{Automatiseerbaar}} \\                                    
    \hline                                                                     
    \textbf{Volledig}    & 22 &  27,50\% & 0  &   0,00\% & 11 &  91,67\% & 12 & 100,00\% \\
    \textbf{Gedeeltelijk}       & 8  &  10,00\% & 0  &   0,00\% & 0  &   0,00\% & 0  &   0,00\% \\
    \textbf{Niet}        & 50 &  62,50\% & 10 & 100,00\% & 1  &   8,33\% & 0  &   0,00\% \\
    \hline                                                                     
    \multicolumn{9}{c}{\textbf{Detecteerbaar}} \\                                       
    \hline                                                                     
    \textbf{Volledig}    & 24 &  30,00\% & 0  &   0,00\% & 0  &   0,00\% & 0  &   0,00\% \\
    \textbf{Gedeeltelijk}       & 5  &   6,25\% & 1  &  10,00\% & 0  &   0,00\% & 0  &   0,00\% \\
    \textbf{Niet}        & 51 &  63,75\% & 9  &  90,00\% & 12 & 100,00\% & 12 & 100,00\% \\
    \hline                                                                     
    \multicolumn{9}{c}{\textbf{Verandering}} \\                                         
    \hline                                                                     
    \textbf{TypeScript}  & 50 &  62,50\% & 3  &  30,00\% & 0  &   0,00\% & 0  &   0,00\% \\
    \textbf{Templates}   & 10 &  12,50\% & 10 & 100,00\% & 0  &   0,00\% & 0  &   0,00\% \\
    \textbf{Testen}      & 16 &  20,00\% & 2  &  20,00\% & 0  &   0,00\% & 0  &   0,00\% \\
    \textbf{JSON}        & 12 &  15,00\% & 0  &   0,00\% & 12 & 100,00\% & 11 &  91,67\% \\
    \textbf{CLI}         & 12 &  15,00\% & 0  &   0,00\% & 11 &  91,67\% & 12 & 100,00\% \\
    \textbf{n.v.t.}      & 10 &  12,50\% & 0  &   0,00\% & 0  &   0,00\% & 0  &   0,00\% \\
    \hline                                                                     
    \textbf{Syntax}      & 24 &  30,00\% & 6  &  60,00\% & 1  &   8,33\% & 0  &   0,00\% \\
    \textbf{Semantiek}   & 38 &  47,50\% & 4  &  40,00\% & 0  &   0,00\% & 0  &   0,00\% \\
    \bottomrule
  \end{tabular}
  \caption[Resultaten deel 2]{
    \label{tab:resultaten-deel-2}Deel 2 van de resultaten van de updater uitgevoerd op de testomgeving.
    Alle rijen buiten \emph{n.v.t.} onder \emph{Verandering} zijn overlappend, zoals besproken in hoofdstuk~\ref{ch:soorten-aanpassingen}.
  }
\end{table}

\clearpage
\section{Besluit}
\label{ch:besluit}

De testresultaten tonen aan dat de updater een meerwaarde biedt in de ondersteuning van het updateproces.
Ondanks dat het verwachte resultaat niet bereikt is, was het nog steeds mogelijk om een vierde van alle aanpassingen automatisch uit te voeren.
De hoge flexibiliteit van onze aanpak maakt het mogelijk om de updater te herconfigureren voor toekomstige updates.
Bovendien is de updater niet gelimiteerd aan het uitvoeren van Angular-updates.
Dezelfde manier van werken kan toegepast worden om meer algemene refactoringen uit te voeren.
Verder kan men de updater configureren om op andere TypeScript-applicaties te werken.
\medskip

Onze aanpak is gericht op het updaten van meerdere enterprise-applicaties.
We zien in dat dit onderzoek niet in alle gevallen een meerwaarde biedt.
De reële tijdswinst van deze aanpak is afhankelijk van hoe de updater gebruikt wordt en door wie.
De updater configureren om één lijn code aan te passen in één enkele applicatie is contraproductief.
Om het meeste uit de updater te halen, moet de tijd voor de updater te configureren kleiner zijn dan de tijd om de update handmatig uit te voeren.
Deze rekensom is afhankelijk van verschillende variabelen.
Voornamelijk de kennis en ervaring van de persoon die de updater configureert.
Iemand met een diepe kennis over de code van het bedrijf zal deze som beter kunnen inschatten dan iemand zonder deze kennis.
\medskip

In totaal waren er 80 verschillende soorten aanpassingen nodig om een Angular-applicatie van v16 naar v20 te updaten.
Zoals omschreven in hoofdstuk~\ref{ch:stand-van-zaken:angular:aanpassingen-tussen-v16-en-v20} \& \ref{ch:soorten-aanpassingen} hadden deze aanpassingen betrekking op verschillende aspecten van het framework.
Verder hebben we een onderscheid kunnen maken tussen het soort aanpassingen, syntactisch of semantisch.
\medskip

Er bestaan meerdere manieren om code automatisch aan te passen.
Dit kan via tools ingebouwd in IDE's, algoritmes om fouten op te sporen of aan te passen, en machine- of deep learning-modellen.
Voor elke manier zijn er verschillende implementaties beschikbaar.
In dit onderzoek legden we de nadruk op de meest voorkomende: zoek- \& vervangfuncties, language-servers en compiler-tooling.
\medskip

Zoek- \& vervangfuncties in combinatie met compiler-tooling zijn gekozen als de meest geschikte manier om in deze casus toe te passen.
Dit onderzoek heeft hiervoor gekozen vanwege de hoge kans op een succesvolle implementatie en de betrouwbaarheid van de output.
Zoek- \& vervangfuncties zijn welbekend en simpel om mee te werken en te implementeren.
Om hun tekortkomingen te compenseren, werd compiler-tooling gebruikt via de TypeScript Compiler API.
Meerdere studies tonen aan dat compiler-tooling werkt op grote schaal.
Verder geeft het ons dezelfde errordetectie van de compiler, waardoor we op een betrouwbare manier bugs kunnen opsporen.
\medskip

In hoofdstuk~\ref{ch:stand-van-zaken:refactoring:known-problems} bespraken we kort wat de gekende problemen waren bij het refactoren van code.
In de context van onze aanpak kunnen we zeggen dat aanpassingen aan semantiek problematisch zijn om te refactoren.
Er is zowel kennis nodig van de werking van Angular als van de applicaties die Angular gebruiken.
In de update van v16 naar v20 blijkt dat 47,50\% van alle aanpassingen betrekking heeft op de achterliggende semantiek van Angular.
Zelfs als een aanpassing betrekking heeft op zowel syntax als semantiek, blijkt dit moeilijk te automatiseren.

\section{Verder onderzoek}
\label{ch:verder-onderzoek}

In de stand van zaken hebben we verschillende automatisatietechnieken besproken.
Een vergelijking van deze technieken voor toepassing in andere casussen kan waardevol zijn.
Wanneer zou een integratie met een language server, of met AI, gepast zijn bijvoorbeeld?
\medskip

Tijdens het schrijven van dit onderzoek is Angular v21 uitgekomen.
Deze versie komt met nieuwe tools om AI beter te integreren in het ontwikkelingsproces.
Voornamelijk beweert het \textcite{AngularV21Announcement2025} dat het AI toelaat om de nieuwste functionaliteiten te gebruiken.
Dit was één van de redenen dat AI niet gekozen werd in dit onderzoek.
Deze tools zijn momenteel nog experimenteel, maar kunnen veelbelovend zijn.

