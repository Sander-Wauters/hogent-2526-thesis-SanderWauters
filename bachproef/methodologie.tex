%%=============================================================================
%% Methodologie
%%=============================================================================

\chapter{\IfLanguageName{dutch}{Methodologie}{Methodology}}%
\label{ch:methodologie}

%% TODO: In dit hoofstuk geef je een korte toelichting over hoe je te werk bent
%% gegaan. Verdeel je onderzoek in grote fasen, en licht in elke fase toe wat
%% de doelstelling was, welke deliverables daar uit gekomen zijn, en welke
%% onderzoeksmethoden je daarbij toegepast hebt. Verantwoord waarom je
%% op deze manier te werk gegaan bent.
%% 
%% Voorbeelden van zulke fasen zijn: literatuurstudie, opstellen van een
%% requirements-analyse, opstellen long-list (bij vergelijkende studie),
%% selectie van geschikte tools (bij vergelijkende studie, "short-list"),
%% opzetten testopstelling/PoC, uitvoeren testen en verzamelen
%% van resultaten, analyse van resultaten, ...
%%
%% !!!!! LET OP !!!!!
%%
%% Het is uitdrukkelijk NIET de bedoeling dat je het grootste deel van de corpus
%% van je bachelorproef in dit hoofstuk verwerkt! Dit hoofdstuk is eerder een
%% kort overzicht van je plan van aanpak.
%%
%% Maak voor elke fase (behalve het literatuuronderzoek) een NIEUW HOOFDSTUK aan
%% en geef het een gepaste titel.

% Wat gebeurt er in dit hoofdstuk?
In dit hoofdstuk maakt het onderzoek een beslissing over hoe de updater ontwikkeld wordt.
Deze beslissing is gebaseerd op de huidige stand van zaken en de noden van het bedrijf, zoals besproken in Hoofdstuk~\ref{sec:onderzoeksdoelstelling}.
Vervolgens wordt een proof of concept voor de updater ontwikkeld, samen met een testomgeving.
We bespreken de verschillende soorten aanpassingen die moeten gebeuren.
Geven een technisch overzicht van hoe de updater en testomgeving werkt.
En tenslotte wordt de effectiviteit van de updater opgemeten op de testomgeving.

\section{Plan van aanpak}
\label{ch:plan-van-aanpak}

% Hoe ziet de updater eruit?
Op basis van de huidige stand van zaken en de noden van het bedrijf, kiest het onderzoek voor het volgende plan van aanpak.
De updater wordt een collectie aan helperfuncties om een TypeScript command line applicatie te maken die onderliggend gebruikmaakt van zoek- en vervangfuncties op basis van regex in combinatie met de TypeScript compiler API.
Met de hulp van deze functies kan de ontwikkelaar programmatisch een updater maken, specifiek voor de nodige aanpassingen.

% Waarom een collectie van helperfuncties?
De nodige aanpassingen aan een applicatie zijn afhankelijk van de Angular versie.
Deze kunnen complex of simpel zijn.
Door de ontwikkelaar programmatisch de updater te laten configureren, is er een hoge flexibiliteit en uitbreidbaarheid van de updater.
De helperfuncties zorgen voor een hoge abstractie, wat toelaat om snel nieuwe updates te automatiseren.
Al dit geschreven in TypeScript, de programmeertaal waar de ontwikkelaars bekend mee zijn.

% Waarom TypeScript compiler API?
Om dit te implementeren wordt van de TypeScript compiler API gebruikgemaakt.
Door te programmeren op de compiler die normaal gebruikt wordt om de applicaties te compileren, krijgen we toegang tot dezelfde error detectie als de compiler.
Dit zorgt ervoor dat we nieuwe bugs snel en accuraat kunnen opsporen.
Verder is de interne werking, op basis van een AST, goed gedocumenteerd.
Tenslotte zijn we niet rechtstreeks gekoppeld aan Angular, enkel aan de programmeertaal.

% Waarom find & replace?
Om de implementatie voor de ontwikkelaar te vereenvoudigen, wordt de TypeScript compiler API gebruikt in combinatie met zoek- en vervangfuncties op basis van regex.

% Waarom een CLI-applicatie?
Angular komt reeds met CLI-tools; door de updater een CLI-applicatie te maken, past het in de huidige workflow.
Verder geeft dit de mogelijkheid om alle commando's samen te voegen in één script om de updater op meerdere projecten te laten uitvoeren.

\section{Angular aanpassingen}
\label{ch:angular-aanpassingen}

\section{Proof of concept}
\label{ch:proof-of-concept}

\subsection{Opzet updater}
\label{ch:proof-of-concept:opzet-updater}

\subsection{Opzet testomgeving}
\label{ch:proof-of-concept:opzet-testomgeving}

\subsection{Updater evaluatie}
\label{ch:proof-of-concept:updater-evaluatie}

