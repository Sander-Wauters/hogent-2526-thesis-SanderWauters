%%=============================================================================
%% Samenvatting
%%=============================================================================

% TODO: De "abstract" of samenvatting is een kernachtige (~ 1 blz. voor een
% thesis) synthese van het document.
%
% Een goede abstract biedt een kernachtig antwoord op volgende vragen:
%
% 1. Waarover gaat de bachelorproef?
% 2. Waarom heb je er over geschreven?
% 3. Hoe heb je het onderzoek uitgevoerd?
% 4. Wat waren de resultaten? Wat blijkt uit je onderzoek?
% 5. Wat betekenen je resultaten? Wat is de relevantie voor het werkveld?
%
% Daarom bestaat een abstract uit volgende componenten:
%
% - inleiding + kaderen thema
% - probleemstelling
% - (centrale) onderzoeksvraag
% - onderzoeksdoelstelling
% - methodologie
% - resultaten (beperk tot de belangrijkste, relevant voor de onderzoeksvraag)
% - conclusies, aanbevelingen, beperkingen
%
% LET OP! Een samenvatting is GEEN voorwoord!

%%---------- Nederlandse samenvatting -----------------------------------------
%
% TODO: Als je je bachelorproef in het Engels schrijft, moet je eerst een
% Nederlandse samenvatting invoegen. Haal daarvoor onderstaande code uit
% commentaar.
% Wie zijn bachelorproef in het Nederlands schrijft, kan dit negeren, de inhoud
% wordt niet in het document ingevoegd.

% \IfLanguageName{english}{%
% \selectlanguage{dutch}
% \chapter*{Samenvatting}
% \lipsum[1-4]
% \selectlanguage{english}
% }{}

%%---------- Samenvatting -----------------------------------------------------
% De samenvatting in de hoofdtaal van het document

\chapter*{\IfLanguageName{dutch}{Samenvatting}{Abstract}}

% - inleiding + kaderen thema
Het Angular-framework vereenvoudigt het ontwikkelingsproces voor het bouwen van dynamische webapplicaties.
Zoals bij de meeste software ontvangt Angular regelmatig updates.
Deze updates zijn noodzakelijk, omdat ze de cyberveiligheid verbeteren.
Het toepassen van dergelijke updates is echter niet altijd vanzelfsprekend.
% - probleemstelling
Angular verwijdert verouderde functionaliteiten uit het framework.
Daardoor moet de broncode van Angular-applicaties aangepast worden.
Dit type updates vindt om de 6 maanden plaats.
Bij meerdere enterprise-applicaties kan de benodigde tijd voor dit updateproces snel oplopen.
Het bedrijf Stater ervaart dit probleem.
Stater is een end-to-end dienstverlener voor zowel hypothecaire als consumentenkredieten.
Zij willen al hun applicaties updaten van Angular v16 naar v20.
De sprong van vier versies betekent dat er vermoedelijk veel aanpassingen in de broncode nodig zijn.
\medskip

% - (centrale) onderzoeksvraag
Om deze reden wil dit onderzoek achterhalen in welke mate de automatisering van het updateproces van Angular v16 naar v20, over meerdere applicaties, de onderhoudstijd voor ontwikkelaars kan verlagen.
% - onderzoeksdoelstelling
Om hierop een antwoord te formuleren, ontwikkelt dit onderzoek een applicatie om het updateproces tee ondersteunen.
Deze applicatie, de \emph{updater}, fungeert als proof of concept.
% - methodologie
Voor de implementatie van de updater maken we gebruik van zoek- en vervangfuncties in combinatie met de TypeScript Compiler API.
De updater kan programmatisch geconfigureerd worden.
Dit biedt aanzienlijke flexibiliteit, waardoor de updater ook bij toekomstige aanpassingen inzetbaar blijft.
\medskip

% - resultaten (beperk tot de belangrijkste, relevant voor de onderzoeksvraag)
Met onze aanpak is het mogelijk om een kwart van alle nodigde wijzigingen in de update van v16 naar v20 te automatiseren.
Daarnaast zijn we niet beperkt tot aanpassingen in Angular.
Onze aanpak werkt op alle TypeScript-code binnen een project.
% - conclusies, aanbevelingen, beperkingen
De voornaamste beperking van de updater is het uitvoeren van wijzigingen aan logic.
Aanpassingen die inzicht vereisen in de werking van de applicatie vormen hierbij een uitdaging.
Om de updater optimaal te benutten, is het aangewezen om enkel syntactische aanpassingen te automatiseren.

