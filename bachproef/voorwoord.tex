%%=============================================================================
%% Voorwoord
%%=============================================================================

\chapter*{\IfLanguageName{dutch}{Woord vooraf}{Preface}}%
\label{ch:voorwoord}

%% TODO:
%% Het voorwoord is het enige deel van de bachelorproef waar je vanuit je
%% eigen standpunt (``ik-vorm'') mag schrijven. Je kan hier bv. motiveren
%% waarom jij het onderwerp wil bespreken.
%% Vergeet ook niet te bedanken wie je geholpen/gesteund/... heeft

Deze bachelorproef vormt het sluitstuk van mijn opleiding Toegepaste Informatica, met als specialisatie Mobile en Enterprise Development.
Tijdens deze opleiding kreeg ik de kans om aan verschillende projecten te werken en deed ik ervaring op met verschillende programmeertalen en frameworks. 
Wat mij daarbij het meest boeide, was het ontwikkelingsproces zelf. 
Om die reden koos ik ervoor om mijn bachelorproef uit te werken rond een technisch onderwerp.
\medskip

De originele probleemstelling komt van mijn co-promotor, Peter De Seranno.
Het probleem was dat er verschillende Angular-applicaties geüpdatet moesten worden naar een nieuwe versie.
Wat mij opviel, was het grote aantal herhalingen in dit updateproces.
Processen die zich herhalen zijn geschikt om te automatiseren.
Zo ontstond het idee om software te ontwikkelen die het updateproces kan ondersteunen.
Dit project was een leerrijke ervaring waarin ik mijn technische kennis kon toepassen. 
Verder has het Angular framework nieuw voor mij.
Het was interessant om hier meer over te leren.
\medskip

Voor dit project kon ik rekenen op de hulp en begeleiding van enkele personen die ik graag wil bedanken.
In de eerste plaats wil ik mijn promotor, mevrouw Irina Malfait, bedanken voor de opvolging en begeleiding van mijn werk. 
Haar gerichte en constructieve feedback hielp mij om deze bachelorproef naar een hoger niveau te brengen.
Daarnaast wil ik mijn oprechte dank uitspreken aan mijn co-promotor, Peter De Seranno, voor zijn technische ondersteuning en het nalezen van mijn werk. 
Zijn input zorgde ervoor dat ik het overzicht behield en dat het onderzoek steeds praktisch en relevant bleef.
\medskip

Tot slot wil ik mijn ouders bedanken voor hun voortdurende steun en begrip. 
Dankzij hun aanmoediging kon ik de overstap maken naar deze studierichting en mijn opleiding met vertrouwen verderzetten.
Met deze bachelorproef hoop ik iets bij te dragen aan het ontwikkelingsproces en de manier waarop we software refactoren en onderhouden.
