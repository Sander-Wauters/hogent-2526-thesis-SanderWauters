%==============================================================================
% Sjabloon onderzoeksvoorstel bachproef
%==============================================================================
% Gebaseerd op document class `hogent-article'
% zie <https://github.com/HoGentTIN/latex-hogent-article>

\documentclass{hogent-article}

\addbibresource{voorstel.bib}

\studyprogramme{Professionele bachelor toegepaste informatica}
\course{Bachelorproef}
\assignmenttype{Onderzoeksvoorstel}

\academicyear{2025-2026}

\title{Proof of concept: De update automatiseren van Angular versie 16 naar versie 20 in de applicaties van een end-to-end kredietdienstverlener.}

\author{Wauters Sander}
\email{sander.wauters@student.hogent.be}

\supervisor[Co-promotor]{De Seranno Peter (Stater, \href{mailto:peter.deseranno@stater.be}{peter.deseranno@stater.be})}

\specialisation{Mobile \& Enterprise development}
\keywords{Angular, Static code analysis, Automatisatie}

\begin{document}

\begin{abstract}

% Wat is het thema?
Een applicatie updaten naar een nieuwe versie van een softwareframework kan veel tijd in beslag nemen, zeker als de applicatie meerdere versies achterloopt.
Hetzelfde updateproces vervolgens herhalen over meerdere applicaties zorgt ervoor dat de onderhoudstijd snel toeneemt.
% Wat is de onderzoeksvraag?
Het doel van dit onderzoek is om de update van het Angular-webframework van versie 16 naar versie 20 in meerdere applicaties te automatiseren, met als doel de onderhoudstijd voor de ontwikkelaars te verlagen.
% Wat is het probleem? 
Framework updates kunnen de performantie en veiligheid van de applicatie verbeteren.
Het is daarom belangrijk om deze tijdig uit te voeren.
Deze updates zijn niet altijd even simpel om toe te passen, zeker als deze gepaard gaan met veranderingen aan de broncode van de applicatie.
In grotere broncodes neemt het proces om alle veranderingen systematisch aan te brengen veel tijd in beslag.
% Wat gaat er gebeuren in dit onderzoek?
Dit onderzoek start met het maken van een oplijsting van alle nodige aanpassingen tussen Angular versie 16 en versie 20.
Vervolgens worden de verschillende manieren om aanpassingen automatisch uit te voeren onderzocht.
% Hoe gaat het onderzoek uitgevoerd worden?
Op basis hiervan wordt een proof of concept ontwikkeld die het updateproces zal automatiseren waar mogelijk.
Deze proof of concept wordt getest in een gecontroleerde omgeving.
% Wat is denk je dat het resultaat is van het onderzoek?
De effectiviteit wordt beoordeeld aan de hand van het aantal nodige veranderingen die het automatisatieproces correct kan opsporen en uitvoeren ten opzichte van het totaal aantal uit te voeren aanpassingen.
Dit onderzoek verwacht dat het 65\% van alle nodige veranderingen kan automatiseren.
% Wat hebben we aan het resultaat?
Het automatisatieproces zal naar verwachting de nodige tijd om de applicaties te updaten verminderen.

\end{abstract}

\tableofcontents


\section{Inleiding}
\label{sec:inleiding}

% Voor wie doen we dit? Wat doet Stater?
Het bedrijf Stater is een end-to-end dienstverlener voor zowel hypothecaire en consumentenkredieten, ze ondersteunen de kredietverstrekker voor de dienstverlening aan consumenten.
% Welke relevante technologie gebruiken ze?
Binnen het bedrijf zijn er verschillende applicatie dat gebruik maken van het Angular framework.
% Wat is Angular?
Angular is een open-source front-end framework, gebaseerd op de TypeScript programmeertaal, dat gebruikt wordt voor de ontwikkeling van dynamische web applicaties \autocite{Cincovic2019}.
% Wat moet er gebeuren? 
Momenteel is Angular versie 20 (v.20) de meest recente stabiele versie.
Binnen Stater maken de applicaties gebruikt van Angular versie 16 (v.16), maar het bedrijf is van plan de applicaties te updaten naar de recentste versie, Angular v.20.

% Waarom moeten de applicaties geupdate worden?
\textcite{Vaniea2016} omschrijven software updates als het introduceren van nieuwe functionaliteiten, de performantie van de applicatie verbeteren en verzekeren dat de software compatibel blijft met nieuwe software en hardware.
Verder omschrijft deze brond dat het up-to-date houden van software cruciaal voor de cyberveiligheid te garanderen.

% Waarom is dit een probleem?
Vanwege het grote verschil in versies zal het updaten van alle applicaties veel tijd in beslag nemen.
% Waarom automatiseren?
Volgends \textcite{Kaur2015} neemt het onderhoudt van een softwareproject gemiddeld 60\% van de kostprijs in beslag.
Een manier om de tijd voor software-onderhoud in te korten is daarom best interesant.
% Wat is de onderzoeksvraag?
Hieruit komt de vraag: In welke mate is het mogelijk om een applicatie in Angular v.16 automatisch te updaten naar Angular v.20?
% In welke deelvragen wordt de onderzoeksvraag opgesplitst?
Voor deze vraag te beantwoorden worden volgende deelvragen geformuleerd:
\begin{itemize}
  \item Wat zijn de veranderingen tussen Angular v.16 en Angular v.20?
  \item Welke van deze veranderingen kunnen automatisch uitgevoerd worden zonder de functionele en niet-functionele vereisten van de applicatie in drang te brengen?
  \item Wat zijn de manieren om code automatisch aan te passen?
  \item Welke manier(en) is meest geschikt voor toe te passen in deze context?
\end{itemize}

% Wat zal gebeuren om de onderzoeksvraag op te lossen?
Gedurende dit onderzoek zal een applicatie ontwikkeld worden dat een software project in Angular v.16 automatisch update naar Angular v.20.
In de rest van dit document wordt naar deze applicatie verwezen als de ``updater''.
% Wat zal de updater doen?
De updater doorloopt de broncode van een applicatie en maakt een oplijsting van alle nodige aanpassingen en tracht de aanpassing zelf uit te voeren indien mogelijk.
% Hoe wordt de updater geëvalueerd?
Voor de effectiviteit van de updater te meten, worden de uit te voeren updates opgedeeld in verschillende categorieën en wordt gemeten hoeveel van de nodige updates automatisch uitgevoerd kunnen worden.

% Wat zal er in de volgende hoofstukken van dit document besproken worden?
In de volgende sectie wordt een kort overzicht gegeven van de huidige stand van zaken binnen het probleem en oplossingsdomein.
Hierna volgt een beschrijving van de methodologie waar de werking en evaluatie van de updater in meer detail beschreven worden.
En tenslotte worden de verwachte resultaten besproken, waarin een inschatting wordt gegeven naar de bevindingen van het onderzoek.

\section{Literatuurstudie}
\label{sec:literatuurstudie}

\subsection{Veranderingen in Angular}

% Wat zijn de aanpassingen die uitgevoert moeten worden?
De \textcite{AngularUpdateGuide2025} geeft ons een uitgebreid overzicht van alle aanpassingen die nodig zijn voor een Angular applicatie van v.16 naar v.20 te updaten. 
Uit deze bron blijkt dat in totaal er 79 verschillende stappen uitgevoerd worden.

De studie door \textcite{Bavota2012} onderzoek welke veranderingen aan code het meeste kans hebben om nieuwe bugs te introduceren.
In deze studie zijn deze verandering onderverdeeld in 4 categorieën: schadelijk, potentieel schadelijk, niet schadelijk en niet geclassificeerd.
Deze studie maakt het mogelijk om een geïnformeerde inschatting te maken naar welke aanpassingen geautomatiseerd kunnen worden zonder onderwachte bijwerkingen te introduceren.

\subsection{Automatisatie process}

% Wat is de meest voor de hand liggende oplossing?
Eén van de meest bekende manieren om code in bulk aan te passen is het gebruik maken van zoek en vervang functies gebaseerd op reguliere expressies (Regex).
Een reguliere expressie is een sequentie van karakters dat een patroon in een tekst omschrijven.
% Wat zijn de problemen bij deze oplossing?
Aangezien dat Regex text gebaseerd is, kan het geen rekening houden met de semantiek van de programmeertaal.
Uit de studie van \textcite{Michael2019} blijken nog enkele problemen bij de implementatie van Regex, namelijk dat het moeilijk leesbaar, vindbaar, valideerbaar en documenteerbaar is.

% Wat is een alternatief?
Om met de semantiek van de programmeertaal rekening te houden kan gebruik gemaakt worden van een abstract syntax tree.
Zoals omschreven door \textcite{Sun2023}, een abstract syntax tree is een data structuur dat de broncode van een applicatie illustreert en rekening houdt met de syntax en semantiek van de programmeertaal.
Dit laat ons toe om een stuk code aan te passen enkel als het in een bepaalde scope zit.

% Is hier al een tool voor?
Herinner dat Angular gebaseerd is op de TypeScript programmeertaal.
Een bestaande tool voor TypeScript dat gebruik maakt van een abstract syntax tree is de TypeScript Compiler API.
De studie door \textcite{Reid2023} onderzoekt hoe de TypeScript compiler gebruikt kan worden voor het corrigeren van foutieve code fragmenten. 
Deze studie raad aan om de TypeScript compiler te gebruiken voor statische code analyse vanwege de effectiviteit, accuraatheid en mogelijkheid om foutieve code te detecteren.

% Wat is een alternatief?
Een alternatief op de TypeScript compiler dat ook gebruikt maakt van een abstract systax tree is het TypeScript Language Server Protocol (LSP).
Het LSP, als omschreven door \textcite{Bork2023}, is een open protocol voor gebruik in verschillende code editors of integrated development environments (IDEs) dat programmeertaal specifieke functies voorziet zoals: automatische code aanvullen en code diagnostiek. 
Dezelfde bron omschrijft LSPs als het de facto standaard protocol voor de implementatie van deze functies in IDEs.

% Wat is een alternatief?
Tenslotte is het mogelijk om AI-tools in te zetten voor deze aanpassingen te maken.
Met de recente opkomst van AI-tools dat specifiek gemaakt zijn voor programmeren is het mogelijk om deze taak uit te besteden aan AI.
Uit de studie door \textcite{Hodovychenko2025} blijkt dat AI gedreven tools een gebrek hebben aan transparantie en risico lopen voor de semantiek van de programmeertaal over tijd fout te interpreteren.

\section{Methodologie}
\label{sec:methodologie}

\subsection{Literatuurstudie}
\label{sec:methodologie:literatuurstudie}

% Wat moet eerst onderzocht worden?
Dit onderzoek start met een uitgebreide literatuurstudie naar de verschillen tussen Angular v.16 en Angular v.20.
% Wat krijgen we uit dit onderzoek?
De nodige verandering worden opgelijst en onderverdeeld in verschillende categorieën.
% Waarom hebben we dit nodig?
Deze lijst zal gebruikt worden voor de capaciteiten van de updater te bepalen.

% Wat moet nog onderzocht worden?
Verder wordt onderzocht wat de verschillende manieren zijn om automatisch code te updaten.
% Wat krijgen we uit dit onderzoek?
Eén of meerdere manieren worden verkozen om te implementeren op basis van volgende parameters:
\begin{itemize}
  \item De complexiteit van de implementatie. Een voorkeur wordt gegeven aan het gebruik maken van bestaande tools over het ontwikkelen van nieuwe algoritme.
  \item De betrouwbaarheid van de output. Is het mogelijk om een correct configuratie een incorrecte output te krijgen?
\end{itemize}

% Hoe lang duurt dit?
Deze literatuurstudie neemt één tot twee weken in beslag en het resultaat dient als basis voor de volgende fase van het onderzoek.

\subsection{Ontwikkeling}

% Wat wordt ontwikkeld?
In deze fase van het onderzoek zal de updater ontwikkeld worden op basis van de voorafgaande literatuurstudie.

% Wat doet de updater?
De updater heeft drie functies: het detecteren van aanpassingen in de broncode, het evalueren of deze aanpassingen automatisch kunnen uitgevoerd worden, en de aanpassingen uitvoeren indien mogelijk.
Elke gedetecteerde aanpassing wordt geregistreerd, waaronder de locatie in de broncode, de aard van de wijziging, de categorie, en of de updater in staat is om de wijziging automatisch uit te voeren.

% Hoe wordt het getest?
De updater wordt in een gecontroleerde omgeving getest om de stabiliteit te verzekeren.
% Hoelang duurt dit?
Voor het ontwikkelen van de applicatie wordt vier tot vijf weken voorzien.

\subsection{Evaluatie}

% Hoe krijgen we resultaten?
De updater wordt uitgevoerd op één van de applicatie binnen Stater.
% Wat doen we met deze resultaten?
Vervolgens wordt het resultaat van de updater geëvalueerd aan de hand van het aantal uitgevoerde aanpassingen ten opzichte van het totaal aantal gedetecteerde aanpassingen.

\section{Verwacht resultaat, conclusie}
\label{sec:verwachte_resultaten}

% Wat zijn de verwachtingen?
Op basis van de literatuurstudie en de gehanteerde methodologie verwacht dit onderzoek dat minstens 75\% van alle nodige aanpassingen automatisch uitgevoerd kunnen worden.
% Wat hebben we hieraan?
Het automatisatie proces zal naar verwachting de nodige tijd voor de applicaties te updaten verminderen.
De hoeveelheid tijd dat in totaal bespaard zal worden zal afhangen van het aantal applicatie en de grootte van de broncode dat geüpdatet moet worden.
Het ontwikkelen van de updater heeft uiteraard ook tijd in beslag genomen.
Deze oplossing zal wellicht meer tijd in beslag nemen als enkel één kleine applicatie geüpdatet moet worden.

% Wat kan nog onderzocht worden?
Of dit de meest effectieve manier is voor deze casus op te lossen is open voor debat.
Verder onderzoek zal uitgevoerd worden naar de performantie, accuraatheid en complexiteit van de verschillende implementaties besproken in de literatuurstudie.



\printbibliography[heading=bibintoc]

\end{document}
