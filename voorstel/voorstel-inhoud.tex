
\section{Inleiding}
\label{sec:inleiding}

% Voor wie doen we dit? Wat doet Stater?
Het bedrijf Stater is een end-to-end dienstverlener voor zowel hypothecaire als consumentenkredieten.
Ze ondersteunen de kredietverstrekker voor de dienstverlening aan consumenten.
% Welke relevante technologie gebruiken ze?
Binnen het bedrijf zijn er verschillende applicaties die gebruikmaken van het Angular-webframework.
% Wat is Angular?
Angular is een open-source front-end framework gebaseerd op de TypeScript programmeertaal voor de ontwikkeling van dynamische webapplicaties \autocite{Cincovic2019}.
% Wat moet er gebeuren? 
Momenteel is Angular versie 20 (v20) de meest recente stabiele versie.
Binnen Stater maken de applicaties gebruik van Angular versie 16 (v16); het bedrijf is van plan de applicaties te updaten naar de recentste versie, Angular v20.

% Waarom moeten de applicaties geupdate worden?
De updates niet uitvoeren is geen optie.
Volgens de studie door \textcite{Vaniea2016} zijn software-updates nodig, omdat het nieuwe functionaliteiten introduceert, de performantie verbetert en verzekert dat de software compatibel blijft met andere nieuwe software.
Verder omschrijft deze bron dat het up-to-date houden van software cruciaal is om de cyberveiligheid te garanderen.

% Waarom is dit een probleem?
Het grote verschil in versies zorgt ervoor dat het updaten van alle applicaties veel tijd in beslag neemt.
Dit is geen eenmalig probleem; volgens \textcite{Callaghan2023} krijgt Angular een nieuwe versie om de 6 maanden.
% Waarom automatiseren?
De studie door \textcite{Kaur2015} beweert dat het onderhouden van een softwareproject gemiddeld 60\% van de kostprijs in beslag neemt.
Een manier om de tijd voor software-onderhoud in te korten is daarom best interessant.
% Wat is de onderzoeksvraag?
Hieruit komt de vraag: In welke mate kan de automatisering van het updateproces van Angular v16 naar v20, bij meerdere applicaties, de onderhoudstijd voor developers verlagen?
% In welke deelvragen wordt de onderzoeksvraag opgesplitst?
Om deze vraag te beantwoorden zijn de volgende deelvragen geformuleerd:
\begin{itemize}
  \item Hoeveel veranderingen moeten uitgevoerd worden om Angular van v16 naar v20 te updaten? 
  \item Wat zijn statistiek gezien de meest voorkomende problemen bij het refactoren van code? 
  \item Welke manieren bestaan om code automatisch aan te passen zonder ongewenste veranderingen uit te voeren? 
  \item Welke manier om code automatisch aan te passen is het meest geschikt om in deze casus toe te passen?
\end{itemize}

% Wat zal gebeuren om de onderzoeksvraag op te lossen?
Gedurende dit onderzoek zal als proof of concept een applicatie ontwikkeld worden die een softwareproject in Angular v16 automatisch updatet naar Angular v20.
In de rest van dit document wordt naar deze applicatie verwezen als de ``updater''.
% Wat zal de updater doen?
De updater doorloopt de broncode van een applicatie en maakt een oplijsting van alle nodige aanpassingen en tracht de aanpassing zelf uit te voeren indien mogelijk.
% Voor wie is dit?
Het doelpubliek van de updater zijn de personen die anders deze aanpassingen manueel uitvoeren.
% Hoe wordt de updater geëvalueerd?
De effectiviteit van de updater wordt bepaald aan het aantal gedetecteerde en opgeloste aanpassingen tegenover het totaal van de nodige aanpassingen.

% Wat zal er in de volgende hoofstukken van dit document besproken worden?
In de volgende sectie wordt een kort overzicht gegeven van de huidige stand van zaken binnen het probleem- en oplossingsdomein.
Hierna volgt een beschrijving van de methodologie, waar de werking en evaluatie van de updater in meer detail beschreven is.
En tenslotte worden de verwachte resultaten besproken, waarin een inschatting wordt gegeven van de bevindingen van het onderzoek.

\section{Literatuurstudie}
\label{sec:literatuurstudie}

\subsection{Uit te voeren veranderingen}

% Wat zijn de aanpassingen die uitgevoert moeten worden?
De Angular update guide door \textcite{AngularUpdateGuide2025} geeft een uitgebreid overzicht van alle aanpassingen die nodig zijn om een Angular-applicatie van v16 naar v20 te updaten. 
Uit deze bron blijkt dat in totaal 80 verschillende stappen uitgevoerd moeten worden.
Deze stappen gaan van het uitvoeren van commando's tot verschillende aanpassingen aan de code.

% Welk soort bestanden gebruikt Angular?
Zoals omschreven in de studie door \textcite{Cincovic2020}, maakt Angular gebruik van 3 verschillende soorten bestanden:
\begin{itemize}
  \item TypeScript-bestanden die de bedrijfslogica bevatten.
  \item HTML-bestanden die de structuur van de UI omschrijven.
  \item CSS-bestanden die de visuele representatie van de UI omschrijven.
\end{itemize}

% Is er een manier voor in te schatten of een aanpassing geautomatiseerd kan worden?
De studie door \textcite{Di2020} onderzoekt welke veranderingen aan code de meeste kans hebben om nieuwe bugs te introduceren.
Deze studie maakt het mogelijk om een geïnformeerde inschatting te maken van welke aanpassingen geautomatiseerd kunnen worden zonder ongewenste bijwerkingen te introduceren.

\subsection{Automatisatie process}

% Wat is de meest voor de hand liggende oplossing?
Eén van de simpelste manieren om code in bulk aan te passen is het gebruikmaken van zoek- en vervangfuncties gebaseerd op reguliere expressies (Regex).
De studie van \textcite{Michael2019} omschrijft Regex als een sequentie van karakters die een patroon in een tekst omschrijft.
% Wat zijn de problemen bij deze oplossing?
Uit dezelfde studie blijken enkele problemen bij de implementatie van Regex, namelijk dat het moeilijk leesbaar, vindbaar, valideerbaar en documenteerbaar is.
Verder kan Regex geen rekening houden met de semantiek van de programmeertaal, aangezien het gebaseerd is op enkel tekst.

% Wat is een alternatief?
Om met de semantiek van de programmeertaal rekening te houden, kan gebruikgemaakt worden van een abstract syntax tree.
Zoals omschreven door \textcite{Sun2023}, een abstract syntax tree is een datastructuur die de broncode van een applicatie illustreert en rekening houdt met de syntax en semantiek van de programmeertaal.
Dit laat ons toe om een stuk code aan te passen, enkel als het in een bepaalde scope zit.

% Is hier al een tool voor?
Herinner dat Angular gebaseerd is op TypeScript.
Een bestaande tool voor TypeScript die gebruikmaakt van een abstract syntax tree is de TypeScript Compiler API.
De studie door \textcite{Reid2023} onderzoekt hoe de TypeScript Compiler API gebruikt kan worden voor het corrigeren van foutieve codefragmenten. 
Deze studie raadt aan om de TypeScript Compiler API te gebruiken voor statische code-analyse vanwege de effectiviteit, accuraatheid en mogelijkheid om foutieve code te detecteren.

% Wat is een alternatief?
Een alternatief voor de TypeScript Compiler API dat ook gebruikmaakt van een abstract syntax tree is het Angular Language Server Protocol (LSP).
Het LSP, als omschreven door \textcite{Bork2023}, is een open protocol voor gebruik in verschillende code-editors of integrated development environments (IDEs) dat programmeertaal-specifieke functies voorziet zoals: automatisch code aanvullen en code-diagnostiek. 
Dezelfde bron omschrijft LSP's als het de facto standaardprotocol voor de implementatie van deze functies in IDE's.

% Wat is een alternatief?
Tenslotte is het mogelijk om artificiële intelligentie in te zetten om deze aanpassingen te maken.
Met de recente opkomst van AI-tools die specifiek gemaakt zijn voor programmeren, is het mogelijk om deze taak uit te besteden aan AI.
Uit de studie door \textcite{Hodovychenko2025} blijkt dat AI-gedreven tools een gebrek hebben aan transparantie en risico lopen de semantiek van de programmeertaal in de loop van de tijd fout te interpreteren.
Verder maakt deze studie de bewering dat voor het maken van dit soort AI-tools er nood is aan een grote hoeveelheid betrouwbare trainingsdata.
Het bemachtigen van deze data is problematisch, vooral als het gaat om code die gebruikmaakt van de allernieuwste updates.

\section{Methodologie}
\label{sec:methodologie}

\subsection{Literatuurstudie}
\label{sec:methodologie:literatuurstudie}

% Wat moet eerst onderzocht worden?
Dit onderzoek start met een uitgebreide literatuurstudie naar de verschillen tussen Angular v16 en Angular v20.
% Wat krijgen we uit dit onderzoek?
De nodige veranderingen worden opgelijst en onderverdeeld in verschillende categorieën.
% Waarom hebben we dit nodig?
Deze oplijsting bepaalt de capaciteiten van de updater.
Verder geeft dit een beter overzicht van welke aanpassingen al dan niet geschikt zijn voor automatisatie.
Tenslotte zal deze oplijsting dienen als maatstaf om de effectiviteit van de updater op te meten.

% Wat moet nog onderzocht worden?
Vervolgens wordt onderzocht wat de verschillende manieren zijn om automatisch code te updaten.
% Wat krijgen we uit dit onderzoek?
Eén of meerdere manieren worden verkozen om te implementeren op basis van de volgende parameters:
\begin{itemize}
  \item Complexiteit van de implementatie. Een voorkeur wordt gegeven aan het gebruikmaken van bestaande tools boven het ontwikkelen van nieuwe algoritmes.
  \item Betrouwbaarheid van de output. Zelfde input verwacht een zelfde output.
  \item Gebruiksvriendelijkheid. Het moet bruikbaar zijn door de persoon die normaal manueel de applicaties updatet.
\end{itemize}

% Hoe lang duurt dit?
Deze literatuurstudie neemt één tot twee weken in beslag en het resultaat dient als basis voor de volgende fase van het onderzoek.

\subsection{Ontwikkeling van de proof of concept}

% Wat wordt ontwikkeld?
In deze fase van het onderzoek zal de updater ontwikkeld worden op basis van de voorafgaande literatuurstudie.

% Wat doet de updater?
De updater heeft als minimum de volgende twee functies: het detecteren van code die aangepast moet worden en de aanpassingen uitvoeren indien mogelijk.
Het detecteren van de code speelt een dubbele rol.
In eerste instantie is het nodig om de aanpassing op de correcte plaats uit te voeren.
Indien de aanpassing niet geautomatiseerd kan worden, zorgt het voor een overzicht van waar alle nodige aanpassingen gemaakt moeten worden.

% Hoe wordt het getest?
De updater wordt in een gecontroleerde omgeving getest om de stabiliteit te verzekeren.
Deze gecontroleerde omgeving bestaat uit een testapplicatie gemaakt in Angular v16.
De testapplicatie bevat een codefragment voor elke vooraf geïdentificeerde stap in het updateprocess.

% Hoelang duurt dit?
Voor het ontwikkelen van de proof of concept worden drie tot vier weken voorzien.

\subsection{Evaluatie}

% Wat doen we met deze resultaten?
De effectiviteit van de updater wordt bepaald aan het aantal gedetecteerde en opgeloste aanpassingen tegenover het totaal aantal aanpassingen.

% Hoe krijgen we resultaten?
Een eerste meting wordt uitgevoerd op de gecontroleerde omgeving die gemaakt is in de proof of concept.
Dit geeft een totaalresultaat voor alle aanpassingen die theoretisch nodig zijn.

Tenslotte wordt een tweede meting uitgevoerd om één van de applicaties binnen Stater.
Dit geeft een resultaat voor alle aanpassingen die praktisch nodig zijn.

\section{Verwacht resultaat, conclusie}
\label{sec:verwachte_resultaten}

% Wat zijn de verwachtingen?
Op basis van de literatuurstudie en de gehanteerde methodologie verwacht dit onderzoek dat minstens 65\% van alle nodige aanpassingen automatisch uitgevoerd kan worden.

% Wat hebben we hieraan?
Het automatisatieproces zal naar verwachting de nodige tijd voor de applicaties te updaten verminderen.
De totale hoeveelheid tijd dat in realiteit bespaard wordt is afhankelijk van verschillende factoren zoals: de kennis van de broncode, de ervaring van de programmeur, de grote van de applicaties, \dots.
Deze oplossing zal wellicht meer tijd in beslag nemen als enkel één kleine applicatie geüpdatet moet worden.

% Wat kan nog onderzocht worden?
Dit onderzoek tracht de beste methode te implementeren voor deze casus op basis van gekende literatuur.
Echter, kan het interessant zijn om andere manieren te implementeren en te vergelijken.
Verder onderzoek kan uitgevoerd worden naar de performantie, accuraatheid en complexiteit van de verschillende implementaties besproken in de literatuurstudie.

