
\section{Inleiding}
\label{sec:inleiding}

Het bedrijf Stater maakt momenteel gebruik van het web application framework Angular voor het maken van verschillende applicaties.
De huidige versie van Angular die binnen Stater in gebruik is, is Angular versie 16.
Op dit moment is de laatste stabiele versie van Angular, versie 20.

Stater wil graag de huidige applicaties updaten naar versie 20.
\textcite{Angular} voorziet tooling voor een applicatie automatisch naar een nieuwe versie te updaten.
Maar volgens dezeflde bron is dit gelimiteerd tot applicaties dat enkel 1 versie uiteenlopen.
Dit betekend dat de update handmatig zal uitgevoerd moeten worden.
Gezien de huidige schaal van de codebase zal dit veel tijd in beslag nemen.
Hieruit komt de vraag: Is het mogelijk om een web applicatie in Angular versie 16 automatisch te updaten naar Angular versie 20.

Voor deze vraag te beantwoorden worden volgende deelvragen geformuleerd:

\begin{itemize}
  \item Wat zijn de veranderingen tussen Angular versie 16 en Angular versie 20?
  \item Welke van deze veranderingen kunnen automatisch toegepast worden zonder de functionele of niet-functionele vereisten van de applicatie in drang te brengen?
  \item Welke manieren besteen er om code automatisch aan te passen?
  \item Wat is de meest geschikte manier om code automatisch aan te passen voor de applicaties van dit bedrijf?
\end{itemize}

Het automatisatie process zal uitgevoerd worden op een Angular applicatie waarvan op voorhand geweten is hoeveel aanpassingen er moeten gebeuren.
De effectiviteid het het process wordt beoordeeld aan de hand van het aantal aanpassingen automatisch uitgevoerd kunnen worden ten opzichte van het totaal aantal aanpassingen.

In de volgende sectie wordt een overzicht gegeven van de stand van zaken binnen het probleem en onderzoeksdomein.
Vervolgens wordt de methodologie van dit onderzoek beschreven.
En tenslotte worden de verwachte resultaten besproken, waarin de bevindignen worden samengevat.

% TODO: add references
\section{Literatuurstudie}
\label{sec:literatuurstudie}

\subsection{Angular versies}

De \textcite{Angular update guide} geeft ons een compleet overzicht van alle aanpassingen die moeten gebeuren voor de applicatie tot versie 20 te updaten. Deze updates kunnen onderverdeeld worden in volgende categoriën:

\begin{itemize}\label{categories}
  \item Packages updaten naar een nieuwe versie.
  \item Toevoegen van nieuwe configuraties.
  \item Vervangen van verouderde functionaliteiten. 
  \item Hernoemen van functies en variabelen.
  \item Verwijderen van verouderde functies en variabelen.
\end{itemize}


\subsection{Automatisatie process}

De eenvoudigste manier om code in bulk aan te passen is het gebruik maken van regex gebaseerde find en replace tools.
Dit is een snelle en makelijke manier om code aan te passen zonder dat er nood is aan speciale tooling.
Regex is text gebaseerd en kan dus geen rekening houden met de semantiek van de programmeertaal.
Als gevolg kunnen aanpassingen ongewenste gevolgen met zich meebrengen.

Een andere manier is abstract syntax tree (AST) gebaseerd refactoren.
Een AST geeft informatie over de semantiek van de programmeertaal.
Dit laat ons toe om een stuk code aan te passen enkel als het in een bepaalde scope zit.
Voor het opstellen van een AST bestaan er verschillende open-source programmas.
Een nadeel aan deze manier is dat de abstract syntax tree op de correcte manier uitgelezen moet worden om correcte aanpassingen te maken.

Nog een alternatief is om gebruik te maken van compiler tooling.
Angular is gebaseerd op TypeScript en moet gecompileerd worden naar JavaScript.
Compilers komen ingebouwd met de kennis van de verschillende scopes in een programmeertaal.
Een nadeel aan deze aanpak is dat het uitbreiden van een compiler een complexe taak is.

Een ander optie is door Language Server Protocol (LSP) integratie.
LSP's komen met programmeertaal specifieke tools zoal: code completion, syntax highlighting, het markeren van warnings en errors, etc.
In het begin van 2020 zijn LSP's de ``norm'' geworden voor de implementatie van inteligente programmeertaal tools.
Deze tools kunnen makelijk uitgebrijd worden door middle van script.
Een nadeel is dat deze aanpak gelimiteerd is aan de functies van de LSP.

Tenslotte is het mogenlijk om AI tools in te zetten voor deze aanpassingen te maken.
Met de recente opkomst van AI tools dat specifiek gemaakt zijn voor programmeren is het mogelijk om deze taak uit te besteden aan deze tools.
Het probleem met deze aanpak is dat AI tools niet 100\% betrouwbaar zijn.

\section{Methodologie}
\label{sec:methodologie}

\subsection{Literatuurstudie}

Dit onderzoek start met een uitgebreide literatuurstudie naar wat veranderd tussen Angular versie 16 en Angular versie 20, en welke verandering geautomatiseerd kunnen worden.
Verder wordt onderzocht wat de verschillende manieren zijn om automatisch code te updaten.
Eén of meerdere manieren worden verkozen om te implementeren op basis van de complexiteit en betrouwbaarheid.
Deze literatuurstudie neemt één tot twee weken in beslag en het resultaat dient als basis voor de volgende fase van het onderzoek.

\subsection{Ontwikkeling}

In deze fase van het onderzoek zal een applicatie ontwikkeldt worden op basis van de voorafgaande literatuurstudie.
Deze applicatie wordt in een gecontroleerde omgeving getest om de stabiliteit te verzekeren.
Voor het ontwikkelen van de applicatie wordt vier tot vijf weken voorzien.

\subsection{Proof of Concept}

Duurende twee weken zal het automatisatie process uitgevoerdt worden op de verschillende applicaties binnen het bedrijf.
Er wordt gemeten op aantal handelingen er automatisch uitgevoerdt worden onderverdeeld per categorie.

\section{Verwacht resultaat, conclusie}
\label{sec:verwachte_resultaten}

Dit onderzoek verwacht, op basis van het literatuuronderzoek en de gehanteerde methodologie, dat op zijn minst 75\% van alle nodige aanpassingen automatisch uitgevoerdt kunnen worden.
Het automatisatie process zal naar verwachting de nodige tijd voor de applicaties te updaten verminderen.


