
\section{Inleiding}
\label{sec:inleiding}

% Voor wie doen we dit? Wat doet Stater?
Het bedrijf Stater is een end-to-end dienstverlener voor zowel hypothecaire en consumentenkredieten, ze ondersteunen de kredietverstrekker voor de dienstverlening aan consumenten.
% Welke relevante technologie gebruiken ze?
Binnen het bedrijf zijn er verschillende applicatie dat gebruik maken van het Angular framework.
% Wat is Angular?
Angular is een open-source front-end framework, gebaseerd op de TypeScript programmeertaal, dat gebruikt wordt voor de ontwikkeling van dynamische web applicaties.
% Wat moet er gebeuren? 
Momenteel is Angular versie 20 (v.20) de meest recente stabiele versie.
Binnen Stater maken de applicaties gebruikt van Angular versie 16 (v.16), maar het bedrijf is van plan de applicaties te updaten naar de recentste versie, Angular v.20.

% Waarom moeten de applicaties geupdate worden?
Software updates introduceren nieuwe functionaliteiten, kunnen de performatie van de applicatie verbeteren en verzekeren dat de software compatibel blijft met nieuwe software and hardware.
Verder is het up-to-date houden van software cruciaal voor de cyberveiligheid te garanderen.

% Waarom is dit een probleem?
Vanwege het grote verschil in versies zal het updaten van alle applicaties veel tijd in beslag nemen.
% Wat is de onderzoeksvraag?
Hieruit komt de vraag: Is het mogelijk om een applicatie in Angular v.16 automatisch te updaten naar Angular v.20?
% In welke deelvragen wordt de onderzoeksvraag opgesplitst?
Voor deze vraag te beantwoorden worden volgende deelvragen geformuleerd:
\begin{itemize}
  \item Wat zijn de veranderingen tussen Angular v.16 en Angular v.20?
  \item Welke van deze veranderingen kunnen automatisch uitgevoerdt worden zonder de functionele en niet-functionele vereisten van de applicatie in drang te brengen?
  \item Wat zijn de manieren om code automatisch aan te passen?
  \item Welke manier(en) is meest geschikt voor toe te passen in deze context?
\end{itemize}

% Wat zal gebeuren om de onderzoeksvraag op te lossen?
Gedurende dit onderzoek zal een applicatie ontwikkeld worden dat een software project in Angular v.16 automatisch update naar Angular v.20.
In de rest van dit document wordt naar deze applicatie verwezen als de ``updater''.
% Wat zal de updater doen?
De updater doorloopt de broncode van een applicatie en maakt een oplijsting van alle nodige aanpassingen en tracht de aanpassing zelf uit te voeren indien mogelijk.
% Hoe wordt de updater geëvalueerd?
Voor de effectiviteit van de updater te meten, worden de uit te voeren updates opgedeeld in verschillende categoriën en wordt gemeten hoeveel van de nodige updates automatisch uitgevoerdt kunnen worden.

% Wat zal er in de volgende hoofstukken van dit document besproken worden?
In de volgende sectie wordt een kort overzicht gegeven van de huidige stand van zaken binnen het probleem en oplossingsdomein.
Hierna volgt een beschrijving van de methodologie waar de werking en evaluatie van de updater in meer detail beschreven worden.
En tenslotte worden de verwachte resultaten besproken, waarin een inschatting wordt gegeven naar de bevindingen van het onderzoek.

% TODO: add references
\section{Literatuurstudie}
\label{sec:literatuurstudie}

\subsection{Angular versies}

De \textcite{Angular update guide} geeft ons een compleet overzicht van alle aanpassingen die moeten gebeuren voor de applicatie tot versie 20 te updaten. Deze updates kunnen onderverdeeld worden in volgende categoriën:

\begin{itemize}\label{categories}
  \item Packages updaten naar een nieuwe versie.
  \item Toevoegen van nieuwe configuraties.
  \item Vervangen van verouderde functionaliteiten. 
  \item Hernoemen van functies en variabelen.
  \item Verwijderen van verouderde functies en variabelen.
\end{itemize}


\subsection{Automatisatie process}

De eenvoudigste manier om code in bulk aan te passen is het gebruik maken van regex gebaseerde find en replace tools.
Dit is een snelle en makelijke manier om code aan te passen zonder dat er nood is aan speciale tooling.
Regex is text gebaseerd en kan dus geen rekening houden met de semantiek van de programmeertaal.
Als gevolg kunnen aanpassingen ongewenste gevolgen met zich meebrengen.

Een andere manier is abstract syntax tree (AST) gebaseerd refactoren.
Een AST geeft informatie over de semantiek van de programmeertaal.
Dit laat ons toe om een stuk code aan te passen enkel als het in een bepaalde scope zit.
Voor het opstellen van een AST bestaan er verschillende open-source programmas.
Een nadeel aan deze manier is dat de abstract syntax tree op de correcte manier uitgelezen moet worden om correcte aanpassingen te maken.

Nog een alternatief is om gebruik te maken van compiler tooling.
Angular is gebaseerd op TypeScript en moet gecompileerd worden naar JavaScript.
Compilers komen ingebouwd met de kennis van de verschillende scopes in een programmeertaal.
Een nadeel aan deze aanpak is dat het uitbreiden van een compiler een complexe taak is.

Een ander optie is door Language Server Protocol (LSP) integratie.
LSP's komen met programmeertaal specifieke tools zoal: code completion, syntax highlighting, het markeren van warnings en errors, etc.
In het begin van 2020 zijn LSP's de ``norm'' geworden voor de implementatie van inteligente programmeertaal tools.
Deze tools kunnen makelijk uitgebrijd worden door middle van script.
Een nadeel is dat deze aanpak gelimiteerd is aan de functies van de LSP.

Tenslotte is het mogelijk om AI tools in te zetten voor deze aanpassingen te maken.
Met de recente opkomst van AI tools dat specifiek gemaakt zijn voor programmeren is het mogelijk om deze taak uit te besteden aan deze tools.
Het probleem met deze aanpak is dat AI tools niet 100\% betrouwbaar zijn.

\section{Methodologie}
\label{sec:methodologie}

\subsection{Literatuurstudie}
\label{sec:methodologie:literatuurstudie}

% Wat moet eerst onderzocht worden?
Dit onderzoek start met een uitgebreide literatuurstudie naar de verschillen tussen Angular v.16 en Angular v.20.
% Wat krijgen we uit dit onderzoek?
De nodige verandering worden opgelijst en onderverdeeld in verschillende categorieën.
% Waarom hebben we dit nodig?
Deze lijst zal gebruikt worden voor de capaciteiten van de updater te bepalen.

% Wat moet nog onderzocht worden?
Verder wordt onderzocht wat de verschillende manieren zijn om automatisch code te updaten.
% Wat krijgen we uit dit onderzoek?
Eén of meerdere manieren worden verkozen om te implementeren op basis van volgende parameters:
\begin{itemize}
  \item De complexiteit van de implementatie. Een voorkeur wordt gegeven aan het gebruik maken van bestaande tools over het ontwikkelen van nieuwe algorithme.
  \item De betrouwbaarheid van de output. Is het mogelijk om een correct configuratie een incorrecte output te krijgen?
\end{itemize}

% Hoe lang duurt dit?
Deze literatuurstudie neemt één tot twee weken in beslag en het resultaat dient als basis voor de volgende fase van het onderzoek.

\subsection{Ontwikkeling}

% Wat wordt ontwikkeld?
In deze fase van het onderzoek zal de updater ontwikkeld worden op basis van de voorafgaande literatuurstudie.

% Wat doet de updater?
De updater heeft drie functies: het detecteren van aanpassingen in de broncode, het evalueren of deze aanpassingen automatisch kunnen uitgevoerd worden, en de aanpassingen uitvoeren indien mogelijk.
Elke gedetecteerde aanpassing wordt geregistreerd, waaronder de locatie in de broncode, de aard van de wijziging, de categorie, en of de updater in staat is om de wijziging automatisch uit te voeren.

% Hoe wordt het getest?
De updater wordt in een gecontroleerde omgeving getest om de stabiliteit te verzekeren.
% Hoe lang duurt dit?
Voor het ontwikkelen van de applicatie wordt vier tot vijf weken voorzien.

\subsection{Evaluatie}

% Hoe krijgen we resultaten?
De updater wordt uitgevoerd op één van de applicatie binnen Stater.
% Wat doen we met deze resultaten?
Vervolgens wordt het resultaat van de updater geëvalueerd aan de hand van het aantal uitgevoerde aanpassingen ten opzichte van het totaal aantal gedetecteerde aanpassingen.

\section{Verwacht resultaat, conclusie}
\label{sec:verwachte_resultaten}

Dit onderzoek verwacht, op basis van het literatuuronderzoek en de gehanteerde methodologie, dat op zijn minst 75\% van alle nodige aanpassingen automatisch uitgevoerdt kunnen worden.
Het automatisatie process zal naar verwachting de nodige tijd voor de applicaties te updaten verminderen.


