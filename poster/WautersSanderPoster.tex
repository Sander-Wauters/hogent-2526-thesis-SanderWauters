%==============================================================================
% Sjabloon poster bachproef
%==============================================================================
% Gebaseerd op document class `a0poster' door Gerlinde Kettl en Matthias Weiser
% Aangepast voor gebruik aan HOGENT door Jens Buysse en Bert Van Vreckem

\documentclass[a0,portrait]{hogent-poster}

% Info over de opleiding
\course{Bachelorproef}
\studyprogramme{toegepaste informatica}
\academicyear{2025-2026}
\institution{Hogeschool Gent, Arbeidstraat 14, 9300 Aalst}

% Info over de bachelorproef
\title{Proof of concept: De update automatiseren van Angular versie 16 naar versie 20 in de applicaties van een end-to-end kredietdienstverlener.}
% \subtitle{Ondertitel (eventueel)}

\author{Wauters Sander}
\email{sander.wauters@student.hogent.be}
\supervisor{Irina Malfait}
\cosupervisor{Peter De Seranno (Stater)}
\specialisation{Mobile \& Enterprise development}
\keywords{Angular, TypeScript, Static code analysis, Automatisatie}
\projectrepo{https://github.com/Sander-Wauters/hogent-2526-thesis-SanderWauters}

\begin{document}

\maketitle

\begin{multicols}{2}

\section*{Samenvatting}

Applicatie ontworpen voor een Angular applicatie van v16 naar v20 te updaten.

\section*{Introductie}

Het Angular-framework vereenvoudigt het ontwikkelingsproces voor het bouwen van dynamische webapplicaties.
Zoals bij de meeste software ontvangt Angular regelmatig updates.
Deze updates zijn noodzakelijk, omdat ze de cyberveiligheid verbeteren.
Het toepassen van dergelijke updates is echter niet altijd vanzelfsprekend.
\medskip

% - probleemstelling
Angular verwijdert verouderde functionaliteiten uit het framework.
Daardoor moet de broncode van Angular-applicaties aangepast worden.
Dit type updates vindt om de 6 maanden plaats.
Bij meerdere enterprise-applicaties kan de benodigde tijd voor dit updateproces snel oplopen.
Het bedrijf Stater ervaart dit probleem.
Stater is een end-to-end dienstverlener voor zowel hypothecaire als consumentenkredieten.
Zij willen al hun applicaties updaten van Angular v16 naar v20.
De sprong van vier versies betekent dat er vermoedelijk veel aanpassingen in de broncode nodig zijn.

\section*{Methodologie}

\begin{center}
  \begin{minted}{ts}
    const project = loadProject();
    project.getSourceFiles().forEach((file) =>
      findNodes(
        file,
        (node) =>
          deepestInstanceOf(node, "async") &&
          node.getKind() !== SyntaxKind.AsyncKeyword,
        (node) => node.replaceWithText("waitForAsync"),
      ),
    );
    await saveProject(project);
  \end{minted}
\end{center}

\section*{Onderzoeksvragen}

\textbf{Hoofdvraag}:

In welke mate kan de automatisering van het updateproces van Angular v16 naar v20, bij meerdere applicaties, de onderhoudstijd voor de ontwikkelaars verlagen?

\textbf{Deelvragen}:
\begin{itemize}
  \item Welke veranderingen moeten uitgevoerd worden om Angular van v16 naar v20 te updaten?
  \item Welke manieren bestaan er om code automatisch aan te passen zonder ongewenste veranderingen uit te voeren?
  \item Welke manier om code automatisch aan te passen is het meest geschikt om in deze casus toe te passen?
  \item Wat zijn statistisch gezien de meest voorkomende problemen bij het updaten van code?
\end{itemize}

\section*{Conclusies}

De testresultaten tonen aan dat de updater een meerwaarde biedt in de ondersteuning van het updateproces.
Ondanks dat het verwachte resultaat niet bereikt is, was het nog steeds mogelijk om een vierde van alle aanpassingen automatisch uit te voeren.
De hoge flexibiliteit van deze aanpak maakt het mogelijk voor de updater te herconfigureren voor toekomstige updates.
Bovendien is de updater niet gelimiteerd aan het uitvoeren van Angular-updates.
Dezelfde manier van werken kan toegepast worden om meer algemene refactoringen uit te voeren.
Verder kan men de updater configureren om op andere TypeScript-applicaties te werken.

\begin{center}
  \captionsetup{type=table}
  \begin{tabular}{l*{5}{|lr}}
    \toprule
    \textbf{Categorie} & \multicolumn{2}{c|}{\textbf{Totaal}} & \multicolumn{2}{c|}{\textbf{TypeScript}} & \multicolumn{2}{c|}{\textbf{Testen}} & \multicolumn{2}{c|}{\textbf{Syntax}} & \multicolumn{2}{c}{\textbf{Semantiek}} \\
    \hline
    \textbf{\#Stappen}  & 80 & 100,00\%  & 50 & 100,00\%  & 16 & 100,00\% & 24 & 100,00\%  & 38 & 100,00\% \\
    \hline                                                      
    \multicolumn{11}{c}{\textbf{Automatiseerbaar}} \\                     
    \hline                                                      
    \textbf{Volledig}   & 22 &  27,50\%  & 10 &  20,00\%  & 3  &  18,75\% & 9  &  37,50\%  & 1  &   2,63\% \\
    \textbf{Deels}      & 8  &  10,00\%  & 8  &  16,00\%  & 0  &   0,00\% & 5  &  20,83\%  & 5  &  13,16\% \\
    \textbf{Niet}       & 50 &  62,50\%  & 32 &  64,00\%  & 13 &  81,25\% & 10 &  41,67\%  & 32 &  84,21\% \\
    \hline                                                      
    \multicolumn{11}{c}{\textbf{Detecteerbaar}} \\                        
    \hline                                                       
    \textbf{Volledig}   & 24 &   30,00\%  & 24 &  48,00\%  & 6  &  37,50\% & 12 &  50,00\%  & 13 &  34,21\% \\
    \textbf{Deels}      & 5  &    6,25\%  & 5  &  10,00\%  & 0  &   0,00\% & 2  &   8,33\%  & 4  &  10,53\% \\
    \textbf{Niet}       & 51 &   63,75\%  & 21 &  42,00\%  & 10 &  62,50\% & 10 &  41,67\%  & 21 &  55,26\% \\
    \bottomrule
  \end{tabular}
\end{center}

\section*{Toekomstig onderzoek}

In de stand van zaken hebben we verschillende automatisatietechnieken besproken.
Een vergelijking van deze technieken voor toepassing in andere casussen kan best interessant zijn.
Wanneer zou een integratie met een language server, of met AI, gepast zijn bijvoorbeeld?
\medskip

Tijdens het schrijven van dit onderzoek is Angular v21 uitgekomen.
Deze versie komt met nieuwe tools om AI beter te integreren in het ontwikkelingsproces.
Voornamelijk beweren ze dat het AI toelaat om de nieuwste functionaliteiten te gebruiken.
Dit was één van de redenen dat AI niet gekozen werd in dit onderzoek.
Deze tools zijn momenteel nog experimenteel, maar kunnen veelbelovend zijn.

\end{multicols}
\end{document}


%\section{Sectie met figuur}
%
% De {\LaTeX} figure-omgeving bepaalt zelf waar een afbeelding komt en dat is meestal niet op de plek in de tekst waar de figure-omgeving gedefinieerd wordt. Als je wilt forceren dat afbeeldingen toch in de flow van de tekst blijven, dan kan je dat zoals hieronder:
% 
% \begin{center}
%   \captionsetup{type=figure}
%   \includegraphics[width=1.0\linewidth]{grail}
%   \captionof{figure}{He hasn't got shit all over him. The nose? Where'd you get the coconuts? What do you mean? We shall say `Ni' again to you, if you do not appease us}
% \end{center}
% 
% Let er wel op dat dit tot problemen met bladschikking kan leiden.
