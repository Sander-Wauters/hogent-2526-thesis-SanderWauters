%==============================================================================
% Sjabloon poster bachproef
%==============================================================================
% Gebaseerd op document class `a0poster' door Gerlinde Kettl en Matthias Weiser
% Aangepast voor gebruik aan HOGENT door Jens Buysse en Bert Van Vreckem

\documentclass[a0,portrait]{hogent-poster}

% Info over de opleiding
\course{Bachelorproef}
\studyprogramme{toegepaste informatica}
\academicyear{2025-2026}
\institution{Hogeschool Gent, Arbeidstraat 14, 9300 Aalst}

% Info over de bachelorproef
\title{Proof of concept: De update automatiseren van Angular versie 16 naar versie 20 in de applicaties van een end-to-end kredietdienstverlener.}
% \subtitle{Ondertitel (eventueel)}

\author{Wauters Sander}
\email{sander.wauters@student.hogent.be}
\supervisor{Irina Malfait}
\cosupervisor{Peter De Seranno (Stater)}
\specialisation{Mobile \& Enterprise development}
\keywords{Angular, TypeScript, Static code analysis, Automatisatie}
\projectrepo{https://github.com/Sander-Wauters/hogent-2526-thesis-SanderWauters}

\begin{document}

\maketitle

\begin{multicols}{2}
\begin{large}

% \section*{Samenvatting}
% 
% Applicatie ontwikkeld om het updateproces van Angular-applicaties te ondersteunen.
% De aanpak werd getest door een Angular-applicatie te upgraden van v16 naar v20.
% De evaluatie toont aan dat meer dan 25\% van alle noodzakelijke aanpassingen automatiseerbaar is.
% Deze aanpak is bruikbaar voor het refactoren van verschillende TypeScript-applicaties.
% De applicatie gebruikt de TypeScript-compiler voor een betrouwbare foutdetectie.

\section*{Introductie}

Het Angular-framework vereenvoudigt het ontwikkelingsproces voor het bouwen van dynamische webapplicaties.
Zoals bij de meeste software ontvangt Angular regelmatig updates, die noodzakelijk zijn om onder andere de cyberveiligheid te verbeteren.
Het toepassen van dergelijke updates is echter niet altijd vanzelfsprekend.
\bigskip

% - probleemstelling
Angular verwijdert verouderde functionaliteiten uit het framework, waardoor aanpassingen aan de broncode nodig zijn.
Dit soort updates vindt plaats om de 6 maanden.
Wanneer er meerdere enterprise-applicaties betrokken zijn, kan de benodigde tijd voor dit updateproces snel oplopen.
Het bedrijf Stater ondervindt dit probleem.
Stater is een end-to-end dienstverlener voor zowel hypothecaire als consumentenkredieten.
Ze willen al hun applicaties updaten van Angular v16 naar v20.
Deze sprong van 4 versies vereist aanzienlijke aanpassingen aan de broncode.

\section*{Methodologie}

Dit onderzoek begon met een literatuurstudie naar bekende methoden om code automatisch aan te passen.
Er werd besloten een \emph{updater} te ontwikkelen op basis van de TypeScript Compiler API.
Door de TypeScript-compiler programmatisch aan te spreken en te koppelen aan zoek- en vervangfuncties, werd het mogelijk om code gericht op te sporen en aan te passen.
\bigskip

Een collectie helperfuncties is ontwikkeld om de implementatie te vereenvoudigen.
Dankzij de programmatische opzet biedt de updater een hoge van flexibiliteit.
De updater is geïmplementeerd als een command-lineapplicatie, waardoor deze voor meerdere projecten kan worden gebruikt.
Het volgende codefragment toont hoe alle instanties van een functie gericht kunnen worden hernoemd doorheen meerdere applicaties.

\bigskip
\begin{center}
  \begin{normalsize}
  \begin{minted}{ts}
    const project = loadProject();
    project.getSourceFiles().forEach((file) =>
      findNodes(
        file,
        (node) =>
          deepestInstanceOf(node, "async") &&
          node.getKind() !== SyntaxKind.AsyncKeyword,
        (node) => node.replaceWithText("waitForAsync"),
      ),
    );
    await saveProject(project);
  \end{minted}
  \end{normalsize}
\end{center}

\columnbreak

\section*{Onderzoeksvraag}

In welke mate kan de automatisering van het updateproces van Angular v16 naar v20, bij meerdere applicaties, de onderhoudstijd voor de ontwikkelaars verlagen?

% Deelvragen:
% \begin{itemize}
%   \item Welke veranderingen moeten uitgevoerd worden om Angular van v16 naar v20 te updaten?
%   \item Welke manieren bestaan er om code automatisch aan te passen zonder ongewenste veranderingen uit te voeren?
%   \item Welke manier om code automatisch aan te passen is het meest geschikt om in deze casus toe te passen?
%   \item Wat zijn statistisch gezien de meest voorkomende problemen bij het updaten van code?
% \end{itemize}

\section*{Conclusies}

De testresultaten tonen aan dat de updater een meerwaarde biedt bij de ondersteuning van het updateproces.
Ongeveer een kwart van alle aanpassingen kan automatisch worden uitgevoerd.
De gekozen manier van werken kan eenvoudig overgenomen worden voor toekomstige updates.
Bovendien is de updater niet beperkt tot het uitvoeren van Angular-updates.
Deze werkwijze kan ook worden toegepast voor meer algemene refactorings.
Daarnaast kan de updater worden ingezet voor andere TypeScript-applicaties.

\bigskip
\begin{center}
  \begin{normalsize}
  \captionsetup{type=table}
  \begin{tabular}{l*{5}{|lr}}
    \toprule
    \textbf{Categorie} & \multicolumn{2}{c|}{\textbf{Totaal}} & \multicolumn{2}{c|}{\textbf{TypeScript}} & \multicolumn{2}{c|}{\textbf{Testen}} & \multicolumn{2}{c|}{\textbf{Syntax}} & \multicolumn{2}{c}{\textbf{Semantiek}} \\
    \hline
    \textbf{\#Stappen}    & 80 & 100,00\%  & 50 & 100,00\%  & 16 & 100,00\% & 24 & 100,00\%  & 38 & 100,00\% \\
    \hline                                                      
    \multicolumn{11}{c}{\textbf{Automatiseerbaar}} \\                     
    \hline                                                      
    \textbf{Volledig}     & 22 &  27,50\%  & 10 &  20,00\%  & 3  &  18,75\% & 9  &  37,50\%  & 1  &   2,63\% \\
    \textbf{Gedeeltelijk} & 8  &  10,00\%  & 8  &  16,00\%  & 0  &   0,00\% & 5  &  20,83\%  & 5  &  13,16\% \\
    \textbf{Niet}         & 50 &  62,50\%  & 32 &  64,00\%  & 13 &  81,25\% & 10 &  41,67\%  & 32 &  84,21\% \\
    \hline                                                      
    \multicolumn{11}{c}{\textbf{Detecteerbaar}} \\                        
    \hline                                                       
    \textbf{Volledig}     & 24 &   30,00\%  & 24 &  48,00\%  & 6  &  37,50\% & 12 &  50,00\%  & 13 &  34,21\% \\
    \textbf{Gedeeltelijk} & 5  &    6,25\%  & 5  &  10,00\%  & 0  &   0,00\% & 2  &   8,33\%  & 4  &  10,53\% \\
    \textbf{Niet}         & 51 &   63,75\%  & 21 &  42,00\%  & 10 &  62,50\% & 10 &  41,67\%  & 21 &  55,26\% \\
    \bottomrule
  \end{tabular}
  \caption*{Resultaten van de updater om een Angular-applicatie te updaten van v16 naar v20.}
  \end{normalsize}
\end{center}

\section*{Toekomstig onderzoek}

Dit onderzoek focust op één van de vele automatisatietechnieken.
Een vergelijking van verschillende technieken en hun toepasbaarheid in andere casussen kan interessant zijn voor toekomstig onderzoek.
Zo kan worden onderzocht in welke situaties een integratie met een language server of met AI gepast is.
\bigskip

Tijdens het schrijven van dit onderzoek werd Angular versie 21 uitgebracht.
Deze versie introduceert nieuwe tools om AI beter te integreren in het ontwikkelingsproces.
Voornamelijk beweert het Angular-team dat het AI toelaat om de nieuwste functionaliteiten te gebruiken.
Hoewel deze tools momenteel nog experimenteel zijn, tonen ze potentieel voor verdere automatisering.

\end{large}
\end{multicols}
\end{document}


%\section{Sectie met figuur}
%
% De {\LaTeX} figure-omgeving bepaalt zelf waar een afbeelding komt en dat is meestal niet op de plek in de tekst waar de figure-omgeving gedefinieerd wordt. Als je wilt forceren dat afbeeldingen toch in de flow van de tekst blijven, dan kan je dat zoals hieronder:
% 
% \begin{center}
%   \captionsetup{type=figure}
%   \includegraphics[width=1.0\linewidth]{grail}
%   \captionof{figure}{He hasn't got shit all over him. The nose? Where'd you get the coconuts? What do you mean? We shall say `Ni' again to you, if you do not appease us}
% \end{center}
% 
% Let er wel op dat dit tot problemen met bladschikking kan leiden.
